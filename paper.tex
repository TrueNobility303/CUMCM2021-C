
\documentclass{my_paper}
\usepackage{ctex}
\usepackage[textwidth=444bp,vmargin=2.5cm]{geometry}%设置页边距
\usepackage{array} %主要是增加列样式选项
\usepackage[dvipsnames]{xcolor}%颜色宏包
\usepackage{graphicx}%图片宏包
\usepackage{subfig}
\usepackage{amsmath}%公式宏包

\usepackage{listings}

\lstset{
	%backgroundcolor=\color{red!50!green!50!blue!50},%代码块背景色为浅灰色
%	rulesepcolor= \color{gray}, %代码块边框颜色
	breaklines=true,  %代码过长则换行
	%numbers=left, %行号在左侧显示
	numberstyle= \small,%行号字体
	%keywordstyle= \color{blue},%关键字颜色
	commentstyle=\color{gray}, %注释颜色
%	frame=shadowbox%用方框框住代码块
	frame=single,
	escapeinside=``    % 代码包含中文
}

\usepackage[T1]{fontenc}    
\usepackage{newtxtext, newtxmath}  %两种使用Times New Roman 字体的方法
\graphicspath{{figures/}} %图片路径

\def\lsb{\left[}
\def\rsb{\right]}
\newcommand{\prob}[1]{\mathbb{P}\lsb #1\rsb} %概率符号

\begin{document}
%----------- 中文摘要 ----------
\newpage

\begin{center}
\lunwenbiaoti

\vspace{2ex}
\zhaiyao
\end{center}

\iffalse
开头段:需要充分概括论文内容,一般两到三句话即可,长度控制在三至五行。

问题一中,解决了什么问题;应用了什么方法;得到了什么结果。

问题二中,解决了什么问题;应用了什么方法;得到了什么结果。

问题三中,解决了什么问题;应用了什么方法;得到了什么结果。

结尾段:可以总结下全文,也可以介绍下你的论文的亮点,也可以对类似的问题进行适当的推广。
\fi 

本文旨在解决生产企业的原材料的订购与决策问题。本文的模型综合了各种手段,针对性地建立了各种统计学习模型,并且基于最优化方法建立了规划问题进行求解。

在问题一中,解决供应商的重要性评级问题。在数据分析和特征工程的基础上,基于集成学习的思想,集成了供应量优先模型、熵权模型和聚类模型,给出重要性的候选供应商用于后续问题的求解。

在问题二中,解决企业订购方案和转运方案的多步骤决策问题。本文建立了加权移动平均模型用于回归供应量和订购量的关系,基于ARIMA、HMM等模型预测了对于转运商未来的转运损耗率。在上述数学模型的基础上,将问题拆解为多步骤决策问题,首先基于剪枝和贪心的算法选择候选供应商,再建立概率优化模型衡量优化中的不确定性因素确定订购方案,最后将转运方案的决策问题转化为0-1规划求解。

在问题三中,解决企业多目标的决策问题。本文首先给出该问题在数学上的优化表达式,之后基于供应商的统计模型的建立以及鼓励稀疏性的范数惩罚策略的运用,将多目标决策问题化归为问题二中的单目标决策问题。

在问题四中,解决企业最大产能的寻求问题。对于这个表面上的优化问题,本文将其转化为存在性判定问题。在由参数估计给出的供应商的概率模型和线性规划模型的基础上,改进问题三的解法,并且使用二分搜索算法计算答案。

对于模型给出的结果,采用蒙特卡罗方法进行模拟,验证了给出结果的可靠性。最后,我们对模型进行了稳健性分析和误差分析,并且进行了模型的优缺点对比和模型推广的可行性探究。

\begin{guanjianci}
集成学习 \quad 加权移动平均 \quad 时间序列 \quad 概率优化 \quad 0-1规划\quad 蒙特卡罗
\end{guanjianci}

%----------- 正文 ----------
%----------- 一、问题重述 ----------
\newpage
\section{一、问题重述}

\iffalse
数学建模比赛论文是要我们解决一道给定的问题,所以正文部分一般应从问题重述开始,一般确定选题后就可以开始写这一部分了。这部分的内容是将原问题进行整理,将问题背景和题目分开陈述即可,所以基本没啥难度。本部分的目的是要吸引读者读下去,所以文字不可冗长,内容选择不要过于分散、琐碎,措辞要精练。
注意:在写这部分的内容时,绝对不可照抄原题!(论文会查重)
应为:在仔细理解了问题的基础上,用自己的语言重新将问题描述一遍。语言需要简明扼要,没有必要像原题一样面面俱到。
\fi 

\subsection{问题背景}
对于生产企业,其用于生产的原材料的订购及运输是企业决策中关键的问题。在满足企业正常生产与运转的前提下,企业应进行明智地决策以合理选择不同的供应商和转运商,从而最大限度地减小企业成本。该问题是极具代表性同时又具有重要实际应用价值的问题。

\subsection{问题重述}
在该问题中,涉及两大重要的环节。一是企业对于不同供应商和转运商的评价体系问题,由于生产企业不可能知道供应商和转运商的所有信息,在此种不完全信息条件下的决策,同上需要生产企业对供应商和转运商的重要特征进行数据分析、特征提取和数学模型的构建,在此基础上建立合理的评价体系从而辅助企业决策。二是企业在决策中的规划问题,在满足企业运转所需要的各项约束的前提下,如何合理地分配资源,选择最优的策略从而使得企业达到其最优解。在上述两大环节中,为了更好地分析数据,把握整体的体系,往往需要建立相应的数学模型。我们建立了时间序列模型、加权移动平均值模型等数学模型以解决问题。

%----------- 二、问题分析 ----------
\section{二、问题分析}

\iffalse
从实际问题到模型建立是一种从具体到抽象的思维过程,问题分析这一部分就是沟通这一过程的桥梁,因为它反映了建模者对于问题的认识程度如何,也体现了解决问题的雏形,起着承上启下的作用,也很能反应出建模者的综合水平。这部分的内容应包括:题目中包含的信息和条件,利用信息和条件对题目做整体分析,确定用什么方法建立模型,一般是每个问题单独分析一小节,分析过程要简明扼要, 不需要放结论。建议在文字说明的同时用图形或图表(例如流程图)列出思维过程,这会使你的思维显得很清晰,让人觉得一目了然。(注意:问题分析这一部分放置的位置比较灵活,可以放在问题重述后面作为单独的一节(见到的频率最高),也可以放在模型假设和符号说明后面作为单独的一节,还可以针对每个问题将其写在模型建立中。)
\fi

\subsection{问题一的分析}
问题一给出了402家供应商的供货特征,要求建立数学模型衡量供应商的重要性,本质上为评价体系的构建问题,解决该问题的关键在于从题目给出的数据中挖掘数据的特征,并且根据统计学习等学科知识定义并且计算供应商的重要性。

对于该问题,首先需要对数据进行可视化分析,把握数据的基本性特征,并且对离群点和异常值进行检查和排除,寻找数据分布的规律。在上述数据分析的基础上,针对数据的特点和问题的痛点,我们提出基于的集成学习模型。集成学习模型的优点在于不同的子模型考虑到了问题的不同方面,从而使得集成学习的结果更加全面,具有更高的可信度;集成学习对于某些特殊情况,具有更高的鲁棒性。根据集成模型,从企业对每家供应商每周的订购量以及每家供应商的供应量出发,模型给出了企业的重要性得分,从而用于企业的规划决策中,便于后续几个问题的解决。我们所提出的集成学习模型共包含了:供应量优先模型,熵权模型,基于不同聚类方法的聚类模型。

\subsection{问题二的分析}
问题二要求解决企业的决策问题,企业的目标是最小化总成本,但直接将总成本作为优化目标较为困难。同时,企业主要需要考虑的决策变量有:供应商的选择、订购方案的选择,转运商的选择,同时对所有决策变量进行优化也是相当困难的。因此在问题二中,考虑逐步进行决策,且在每一步决策中仅考虑某一个较为易于求解的子目标进行优化,而非优化最终的总成本。

具体来讲,主要分为以下几个子决策:

\begin{enumerate}

\item 候选供应商的选择。考虑企业在现实决策中,不可能将所有的供应商列入考虑范围内,企业会通常在供应商的重要性的基础上(该部分可以基于问题一的解),对剩余的供应商进行规划。在企业的乐观估计下,只需选择最少的供应商就可以满足企业的生产运转要求,这也是该问题中所考虑的。该子决策任务的优化目标应为最小化候选供应商的数量,使用最少的供应商使得其生产的需求得到相应的满足。

\item 订购方案的制定。在上一步决策的基础上,已经划定了候选供应商的范围,因此后续的决策任务只需在该较小的范围内进行考虑。该简化基于现实决策的理解,同时也使得该数学模型更易于求解。该子决策任务,目的在于为企业决策未来24周的订购方案选择,其优化目标应为:在满足企业生产的约束前提下,使得企业的运转最为经济(原材料订购成本和储存成本的总和最小)。

\item 转运方案的制定。基于企业的订购方案,该步决策任务为为每一周的供应商选择相应的转运商,其中要满足企业的基本原则性条件:每周的每家供应商尽量由一家转运商转运。优化目标为:最小化总材料的损耗量。

\end{enumerate}

经过上述转化,一个复杂的规划问题,被拆解为几个易于求解的小的规划问题。但在每一步的规划中,由于企业的不完全信息和现实生活中的不确定因素,企业需要建立相应的数学模型填补其信息空白。主要有以下几个部分:

\begin{enumerate}

\item 订购量和供应量的相关性模型。对于企业的订单,供应商的满足率在1的附近波动,因此订购量和供应量之间的相关性模型的构建非常重要。对此,我们建立基于加权移动平均的回归模型。同时,基于回归模型,给定未来24周的订购量,可以给出其供应量的预测结果,用于企业的决策过程中。

\item 转运商的损耗率模型。转运商的损耗率是一个重要的不确定因素,在附件中给出了历史5年中每一家转运商的损耗率,对于这一明显具有时序信息的序列,基于不同转运商的不同特点的基础上。我们分别建立了差分整合移动平均自回归模型(ARIMA)、隐马尔可夫模型(HMM)、指数分布逼近模型对其未来的转运的损耗率进行相应的预测。

\end{enumerate}

建立了上述几个数学模型,我们成功地量化了决策中难以把握的不确定性因素,将其数学化。并且将上述模型纳入规划的求解当中,使得求解的优化结果综合考虑了各种不确定性因素,因此所给出的结果将具有更高的鲁棒性。加入不确定性因素后,采用的规划模型也由确定性模型修改为概率优化模型。

\subsection{问题三的分析}

在问题二中,已经建立了详尽的数学模型和行之有效的解决方案。在问题三中考虑化归为问题二进行求解。问题三并没有要求企业逐步进行决策,因此为了进行整体的分析,首先针对其建立端到端的多目标的规划模型。但在实际求解中,考虑到转运商的损耗与供应商的供货量独立的假设,仍然可以沿用问题二中多步骤规划的求解思路。

相较于问题二,问题三有以下几个关键点,下面展开叙述并且给出我们的解决思路:

\begin{enumerate}

\item 问题三为多目标规划,而非问题二的单目标规划。考虑将多目标规划问题转化为加入范数惩罚的单目标规划模型。与问题二相似,供应商的选择应该越少越好,因此采用鼓励稀疏性的1范数惩罚。

\item 问题二建立在选择候选供应商的基础上进行规划求解,此时选择出的候选供应商的特征单一,而问题三供应商的可选范围为所有的402家供应商,如果单纯沿用问题二的模型不能刻画不同供应商不同的供应特点。针对上述观察,需要改进问题二所定义的概率优化模型。

\item 与问题二相比,问题三的求解复杂度更高。完全沿用问题二的求解方法不一定能在有限时间内求出精确解。解决的思路有:一是基于合理的假设更改问题二中的某些限制条件从而降低规划模型的求解复杂度,二是更改求解方法,寻求误差范围内的近似最优解。

\end{enumerate}


\subsection{问题四的分析}

问题四假设企业的产能有提升的潜力,首先要求求解企业可以提升的最大产能。分析问题可以发现,企业的产能并不是可以无限制地提升的,企业需要考虑供应商和转运商的实际情况。对此,需要对供应商和转运商建立其数学模型,从而估计出供应商和转运商的潜力,在此基础上确定企业的产能的提升的潜力。建立起企业决策规划模型后,企业的最大产能必须保证建立的规模模型存在可行解。

在对所有供应商建立概率统计模型的基础上,我们得到了所有402家供应商的供应潜力的估计,据此使用二分搜索的算法得到企业可以提升的最大产能。问题四依托于问题三,但相比起问题三,对于产能较大的企业,其决策模式不能沿用之前产能较低的情况,因此问题四的一大难点对于问题三中的规划问题需要进行修正。

%----------- 三、模型假设 ----------
\section{三、模型假设}

\iffalse
以下是6类常见的模型假设:
1.题目明确给出的假设条件
2.排除生活中的小概率事件(例如黑天鹅事件、非正常情况)
3.仅考虑问题中的核心因素,不考虑次要因素的影响
4.使用的模型中要求的假设
5.对模型中的参数形式(或者分布)进行假设
6.和题目联系很紧密的一些假设,主要是为了简化模型
\fi 

\begin{enumerate}

\item 假设转运商的损耗与供货量独立。根据该假设,可以将订购方案和转运方案的决策看作是两个相对独立的过程。
\item 假设同一家供应商或转运商的统计模型相对稳定。根据该假设,对供应商或者转运商建立相应的统计模型之后,在企业决策的过程中可以将其看作恒定的模型。
\item 每家供应商总是在其能力范围内尽可能地满足企业的订购需求。该假设符合现实生活中供应商盈利的本质,也使得后续的建模过程更为简单。
\item 无特定说明,默认随机变量满足正态分布。如:通常假设残差满足均值为0的正态分布。

\end{enumerate}

%----------- 四、符号说明 ----------
\newpage
\section{四、符号说明}
%使用三线表格最好~
\begin{table}[htbp]%htbp表示的意思是latex会尽量满足排在前面的浮动格式,就是h-t-b-p这个顺序,让排版的效果尽量好。
    \centering
    \begin{tabular}{p{2.0cm}<{\centering}p{9.0cm}<{\centering}p{2.0cm}<{\flushright}}
 %指定单元格宽度, 并且水平居中。
    \hline
    符号 & 说明 & 单位/公式 \\ %换行 
    \hline
    $ O_{i}(t) $ & 企业对第i家供应商第t周的订购量 &  $ m^3 $ \\ 
    $ S_{i}(t) $ & 第i家供应商第t周的供货量 &  $m^3$ \\ 
    $ R_{i}(t) $ &  供货量与订购量之间的残差项 & $S_i(i) - O_{i}(t)$  \\  
    $ T_{i}(t) $ & 第i家转运商第t周所需转运的供应量 &  $m^3$ \\ 
    $ T'_{i}(t) $ & 第i家转运商第t周的被企业的接收量 & $m^3$ \\
    $ \alpha_{i}(t) $ & 第i家转运商第t周的损耗率  & $\frac{T_{i}(t) - T'_{i}(t)}{T_{i}(t)}$ \\
    
    $ c $ &  企业每周所需要的产能 & $m^3$ \\
    $ p_{i} $ & 第i类原材料的采购单价 $(i=1,2,3)$ & 元/$m^3$ \\
    $ q_{i} $ & 第i类原材料的运输单价 $(i=1,2,3)$ & 元/$m^3$ \\
    $ r_{i} $ & 第i类原材料的储存单价 $(i=1,2,3)$ & 元/$m^3$ \\
    $ t $ & 每家转运商每周的转运能力$(t=6000)$ & $m^3$ \\
    
    \hline
    \end{tabular}
\end{table}

\iffalse
本部分是对模型中使用的重要变量进行说明,一般排版时要放到一张表格中。
注意:第一:不需要把所有变量都放到这个表里面,模型中用到的临时变量可以不放。第二:下文中首次出现这些变量时也要进行解释,不然会降低文章的可读性。
\fi 

%----------- 五、模型的建立与求解 ----------

\iffalse
\subsubsection{模型的建立}
模型建立是将原问题抽象成用数学语言的表达式,它一定是在先前的问题分析和模型假设的基础上得来的。因为比赛时间很紧,大多时候我们都是使用别人已经建立好的模型。这部分一定要将题目问的问题和模型紧密结合起来,切忌随意套用模型。我们还可以对已有模型的某一方面进行改进或者优化,或者建立不同的模型解决同一个问题,这样就是论文的创新和亮点。
\subsubsection{模型的求解}
把实际问题归结为一定的数学模型后,就要利用数学模型求解所提出的实际问题了。一般需要借助计算机软件进行求解,例如常用的软件有MATLAB, Spss, Lingo, Excel, Stata, Python等。求解完成后,得到的求解结果应该规范准确并且醒目,若求解结果过长,最好编入附录里。(注意:如果使用智能优化算法或者数值计算方法求解的话,需要简要阐明算法的计算步骤)
\fi 

\section{五、基于集成学习的供应商重要性度量模型(问题一)}

\subsection{数据预处理}
附件一中给出了每一家供应商前五年(240周)内每周的订购量 $O$ 和供应量$S$。且由附件中的表格可以看出,每一家供应商负责供应ABC三类原材料中的的一种,根据题目条件,ABC三类原材料具有不同的效用(单位体积的原材料对于差产能的贡献),其具体数值如下表所示:

\begin{table}[h]
\centering
\begin{tabular}{cc}
\hline
    原材料类型 &   企业每单位产能所消耗的材料$(m^3)$ \\
\hline
 A & 0.6 \\
 B & 0.66 \\
 C & 0.72 \\
\end{tabular}
\end{table}

给定企业每单位产能所消耗的材料$m$,则ABC三类原材料的效用$u$可以定义为其倒数$u = \frac{1}{m}$ , 此时ABC三类原材料的效用比为$ \frac{1}{0.6} : \frac{1}{0.66} : \frac{1}{0.72} $。对于附件中给出的订购量和供应量,实际上需要根据其不同材料的效用比作为比例系数,得到实际上该订购量和供应量对于产能的等效贡献值。也即使用如下公式:$$ \hat{S} = u \times S  $$ $$ \hat{O} = u \times O $$
得到等效后的订购量$\hat{O}$和供应量$ \hat{S}$.

\subsection{数据分析}

首先绘制图表,可视化对应的数据,把握数据的规律。该数据最主要的规律为稀疏性,大多数供应商的订购量和供应量都为0值。以供应量为例,总共有$240 \times 402 = 96480$条数据,但其中仅有25784 的非零元素,非零元素占总元素的比值仅有 14.2 \%. 造成上述稀疏性的同时有客观上和主观上的原因。其主要的客观原因在于:重要的企业数目并非很多,且企业仅需要相对于可供选择的供应商总数较少的供应商便可以满足其正常生产运转的需求。在主观层面,考虑实际生活,该稀疏性也可以找到很多合理的解释,例如:生产企业通常会和供应商进行一定的长期合作关系,在一段时间内企业会选择较为固定的几家供应商。

\subsection{特征工程}
附件一中给出了每一家供应商前五年(240周)内每周的订购量 $O$ 和供应量$S$。在此基础上提取以下几个特征,且提取的理由如下所述:

\begin{enumerate}

\item 平均供应量:$MEAN(\hat{S})$, 供应商的供应量是供应商重要性的直接衡量,对于更重要的供应商,企业会加大其订购,从而在一般情况下也会具有较大的供应量。而不用平均订购量衡量该指标的原因是,存在某些供应商由于自身能力等限制,尽管企业加大其订购量,但供应商却不能提供相应的供应量。考虑到该部分供应商的存在,采用每周的平均供应量衡量可以综合考虑到企业对供应商的重视程度以及供应商自身的响应能力。

\item 平均满足率:$MEAN(r),r = \frac{\hat{O}}{\hat{S}}$, 其中满足率定义为实际供应量与企业订购量的比例,含义为供应商对于企业的订购的满足率,由于数据中供应量可能大于订购量,也可能小于订购量,该满足率可能小于1,也可能大于1。较优的满足率应该维持在1的附件,甚至绍大于1,说明企业的订单可以很好地得到满足。在对满足率进行特征提取的时候,对于$S=0$的情况,设置其相应的满足率$r$为特殊值NAN不予考虑,不考虑在均值、方差等统计量的计算之中。

\item  稳定性:$s = 1- VAR(r)$, 衡量供应商的稳定性,一个好的供应商应该对于企业维持稳定的供应,也即供应商的满足率 $r$应该维持在一个相对较高且波动较小的水平,这样的供应商会更加容易得到企业的长期信任。而稳定性用方差与1的差衡量,方差越小说明该值越接近1,稳定性越高。

\end{enumerate}

\subsection{供应量优先模型}

对于上述构建的三个主要特征,显然第一个特征:平均供应量$MEAN(\hat{S})$是衡量企业重要性的最主要特征。如果没有较高的平均供应量作为保证,其他的两个特征就算再理想对不应使得企业的重要性更高。例如:有些企业常年保持在极低的供应量(接近于0),此类企业显然是极不重要的企业,尽管其供应的稳定性很高。同理,就算某些企业具有更高的满足率,但如果本身的供应量就很低,满足率也没有其应有的意义。

根据上述分析,构建供应量优先模型。首先,在仅考虑供应量$\hat{S}$的前提下,对所有的402家供应商进行排序,绘制散点图后可以发现数据的规律:大部分的供应商的平均供应量都接近于0,该部分供应商为不重要的企业;而少部分供应商的
平均供应量很大,说明该部分供应商为很重要的企业;同时,不同供应商的平均供应量之间存在几个差距比较大的明显的分隔阈值,可以根据几个明显的分隔阈值将供应商分为几类。

绘制箱线图和散点图观察数据的规律,可以发现存在着几个区分较为明显的阈值$\tau$。 为了寻找能够合理地区分出50家较为重要的企业的阈值$\tau$,应该在$\frac{50}{402} = 12.4 \%$ 分位数的左右选取。根据观察法,可以选择$\tau =10000$ 的时候,接近$12.4 \% $ 分位数,且此时可见为一个明显的区分点。该区分点也可以通过调用大津算法(OSTU \cite{gonzale2002digital})计算得到,其结果也在该值附近,说明该值的选取较为合理。OSTU算法是数字图像处理中二值化问题的常见算法,在企业重要性的评价问题中,如果将重要和不重要的企业的区分视作一个二分类问题,本质上也可以视作一个二值化问题,因此使用OSTU算法也具有其内在的合理性。在该基准阈值的基础上,综合考虑阈值$\tau$附近的供应商的其他特征,并且对阈值进行微调,当$\tau=14000$的时候,可以选取出50家较为重要的企业,而其他企业为较不重要的企业。

上述模型称为供应量优先模型,可以给出一个较为具有参考意义的基准答案。该模型给出的50家企业序号为: [ 78 208 189 292 273  74 114  3 218 210 244  86 294  80 346  55 367 364 338  40 365  31 284 126 374  37 247 395 307 201 143 194 352 348 306 268 356 330 308 131 139 340 329 275 151 282 108 361 140 229], 其中根据平均供应量的大小升序排列。

\subsection{熵权模型}

供应量优先模型的合理性如上所述,同时其也给出了较为令人信服的结果。但供应量优先模型的弊端在于对于除了平均供应量以外的其他特征,考虑得很少。因此可以考虑使用熵权模型,对不同的特征赋予不同的权重,通过加权得到最终的得分,而权重采用信息熵计算。当某个特征的信息熵更大,说明其包含了更多的信息,该特征对于该分类任务的重要性就会更大。

使用熵权模型,计算得到特征工程中所提取的三个特征的权重分别为: [0.21196796, 0.37585979, 0.41217225]. 得到权重后,需要对特征进行归一化,后根据权重进行加权组合得到最终表示重要性的指标。对该权重进行分析可以发现,熵权模型给出的权重对于本应较为重要的第一个特征$\hat{S}$给出了相对较低的权重,主要是由于数据的稀疏性造成的,因为平均供应量有大量零元素或者接近零元素的存在,导致其数据的二极化较为严重,从而该特征的信息熵却反而较小。

利用上述的熵权模型,可以给出50家重要性较大的企业为:[ 76 129 189 307 201  98   3  86 338 348 291 114 314 150   7 123  55 244 139 346  80 367  40 294 218 364 140 143 365  31 247 284 266 374 308 194 352 330 356 306 268 131 151 340 329 275 282 108 361 229],其中同样按照分值升序排列(分值越大表明重要性越高)。

熵权模型所给出的50家供应量,与简单使用供应量优先模型给出的50家供应商的结果,总有41家完全相同的企业,且这些企业大多为重要性较高的企业。说明两种模型都有其一定和合理性和一致性,从某种意义上说明方法的有效性。

\subsection{聚类分析模型}

由于重要的企业通常和企业有更紧密的联系和合作,这种正反馈机制下,重要和不重要的企业的差异会被拉大,从而导致不同重要程度的企业之间的特征差异较为明显。根据上述观察和分析,可以考虑对数据进行聚类分析。

首先,采用层次聚类法\cite{james2013introduction},从下而上地把小的类合并聚集,聚类过程中得到一棵树形结构。考虑采用两种不同的层次聚类的距离定义方式,average-linkage(即计算两个聚簇中各自数据点的两两距离的平均值)和 complete-linkage(即选择两个最远的一对数据点的距离作为类的距离,两种方式的层次聚类,选择聚簇数为4,都得到了相同的聚类结果:[108 131 139 143 151 194 201 268 275 282 306 307 308 329 330 340 348 352 356 361 395 140 229]. 并且观察上述结果,发现上述聚类给出的23家企业都同时在熵权模型和供应量优先模型所给出的答案中,因此该聚类的确将非常重要的企业聚成了一类。虽然上述聚类结果并不能直接得到50家重要的企业解决该问题,但却可以作为一个重要的参考,同时聚类结果给出的23家企业应当是优先度很高的极为重要的企业。

再者,考虑到层次聚类法,需要手动选择阈值,而考虑使用基于迭代方法的K-Means聚类法。同样选取聚簇数为4,可以得到较重要的企业在某一类中,选取该类结果,得到43家供应商:[  3   7  31  40  55  80 108 114 123 131 139 140 143 150 151 194 218 229 244 247 266 268 275 282 284 291 294 306 308 314 329 330 338 340 346 348 352 356 361 364 365 367 374]。该43家供应商同样与之前模型给出的所有结果都有较大的交集,说明K-Means所得到的结果也是某种意义上合理的。例如:K-Means算法给出的43家供应商,全部落在熵权模型所给出的50家供应商之中,说明此部分企业的确较为具有区分度。

\subsection{集成学习模型}

根据上述的分析,上述几个模型尽管在某种意义上有很大的一致性,但都各有其优缺点。考虑到上述情况,建立集成学习模型,由几个较弱的分类器出发,基于Bagging的方式得到一个较强的分类器。本节的代码采用R和Python完成,其中统计机器学习相关的Python包使用了sklearn,代码详见附录。

采用集成学习模型得到最终的50家重要供应商为:[126  98  37   3  86 291 395 338 314 307 201 114 150   7 123 244  80 266 218  55 294 346 367 364  40 348 365  31 284 374 247 143 194 352 306 356 268 139 308 330 131 340 329 151 275 282 108 140 361 229]。

\section{六、基于统计模型和最优化方法的多步骤企业规划(问题二)}

\subsection{候选供应商的选择}

在企业的乐观估计下,只需要选择最少数目的供应商就可以满足企业的需求。该决策问题的困难点在于,企业并不知道供应商的供应潜力,因此企业需要基于问题一中对供应商的供货特征进行分析的结果进行评估。而上述结果已经反映在问题一所选出的50家较为重要的供应商中,因此在该决策任务,也仅限定在上述50家企业中进行选择。

\subsubsection{基于剪枝和贪心算法的候选供应商选择}

考虑问题的限制条件,可以给出一些剪枝条件,作为一种过滤的手段,过滤出困可能出现在候选解的供应商,从而降低该问题的复杂度。

\begin{table}[h]
\centering
\begin{tabular}{cc}
\hline
比对项 & 比例关系(A:B:C) \\
\hline
效用 & $\frac{1}{0.6} : \frac{1}{0.66} : \frac{1}{0.72}$ \\
订购成本 & 1.2: 1.1 : 1.0  \\
运输储存成本 & 1: 1: 1\\

\end{tabular}
\end{table}

\begin{enumerate}


\item 原材料类型的优先级。考虑ABC三种原材料的效用、订购、运输、储存成本可以发现,其具有如上表所示的关系。通过简单的计算和对比可以发现,在ABC三种原材料中,对于订购成本该项的贡献,A和C是等价的,且AC两种原材料显然优于材料B。因此在考虑选择不同的供应商的时候,应该优先考虑选择供应A或C这两类原材料的供应商。因此,可以优先排除性价比低的B类供应商,而在A和C两类供应商之间,由于其性价比相同但是单位产能所需要的A类原材料低于C类原材料,也应该优先考虑A类供应商。

\item 最大转运限制。根据转运决策的要求,每一家供应商所供应的原材料应该尽可能由同一家转运公司所承担。因此,该转运决策的限制实际上也限制了供应商的供应限制,也不应该超过上述每周6000$m^3$转运限制。由于订购量和供应量密切相关,因此有理由认为,企业决策订购方案的时候,可以加上最大订购量不超过6000$m^3$的限制条件。

\item 供应潜力限制。分析供应商5年之内的供货特征,可以发现很多供应商并不具备大量供货的潜力。考虑用供应商5年内的最大历史供货量作为其供货潜力的一个衡量,若其最大历史供货量不能超过6000$m^3$,则有理由继续认为其在未来24周内也不可能进行如此大规模的供货。考虑该供应潜力的限制,可以利用其作为剪枝筛选掉最大历史供货量不超过6000$m^3$的企业。该剪枝的理由有两方面:一方面,此类公司不具备较大的供货潜力,可以说明其通常不是重要的公司。令一方面,实验结果证明,在上述剪枝之在剩余的公司中选择仍然可以满足企业的需求(存在可行解),因此上述剪枝并不会对解造成过多的影响。

\end{enumerate}

基于上述观察,可以提出适用于该决策任务的贪心算法:在经过剪枝算法过滤掉部分供应商之后,在剩余的供应商中预先计算得到对每一家供应商进行最大订购时该供应商对总体优化目标产生的代价,每次贪心地选择产生代价最小而同时更重要的的一家供应商,直到所选择出来的供应商可以维持企业的生产需求为止。将上述贪心算法给出的企业数目,作为最小满足条件的企业数目,给出的解为:[201, 307, 395],可以看出给出的都为A类原材料的供应商。

\subsubsection{基于加权移动平均值的候选供应商选择}

上述基于剪枝和贪心算法的优点是算法实现简单且效率较高,但其缺点是剪枝和贪心的过程中对实际情况做了较多的假设。例如:上述贪心算法并没有考虑到供应商满足率的不确定性问题:供应商所提供的原材料不一定正好是企业的订购量。同时,贪心算法中简单的使用企业的历史最大供货量定义其供货潜力在某种程度上时过于乐观的,比如该算法所生成的解中包含了S201这家供应商。但观察该供应商的历史供货特点可以发现,该供应商对于企业并不是持续稳定的供货,而是在某几个少数时间节点上面存在极大宗的供货(30000$m^3$的供应量)。贪心算法的解将该供应商纳入可行解中,忽略了该供应商巨大的不确定性因素,某种程度来说,对于企业(特别是决策者为风险厌恶类型的企业)是不利的结果。

基于上述考虑,此问题中我们使用加权移动平均值来建立每个供应商的订货量-供货量曲线关系,并进一步计算具体的供应商选择。据观察题中给定的$O_i(t)$与$S_i(t)$数据点,我们认为时间变量$t$的影响可以忽略,$(S_i,O_i)$可以认为是离散的数据点;其近似满足恒等关系,但不同供应商在各自不同的区间上会出现严重的供货能力下降。对于某一给定的企业$i$,其收到订购量$o$后,供货量取为

$$ S_i(o) = \frac{ \sum_{i=1}^{402} S_i w_i}{ \sum_{i=1}^{402}w_i } $$
其中权值$w_i$的计算方法为:

$$ w_i(o) = \operatorname{exp} ( - \frac{(O_i-o)^2}{\sigma^2}  ) $$

其中,$\sigma$控制距某位置$o$同一距离的数据点对$o$位置的影响。较大的$\sigma$值使曲线变得平滑,但将脱离实际数据点;较小的$\sigma$使曲线贴近实际数据点而导致全局的不光滑。此处$\sigma$取为$500$,对于距$o$点$1000$单位的数据点,相应权值小于2\%。

该模型给出了一个有实用价值的建立$S(O)$关系的方法,它有如下特征:
\begin{enumerate}
\item 对于数据点稠密的、线性关系明显的区域,算法效果近似线性插值,表示供应情况正常,供应量基本满足订购量。
\item 对于数据点稀少的区域,由于距有效数据点距离较大,值近似为0,表示供应商无力承担此订购量。
\item 在数据点分布突变、以及整体数据点稀疏分布的情况下,算法给出的预测值相对于给出供货情况的一种合理推测,包括一段区域内行为的适量外延,或模拟供应情况正常时的小波动。
\item 在供应量在近似于0和订货量之间有较大波动的情况下,我们认为模型给出的值表示特定订货量下,供应商供货能力的一种期望。
\end{enumerate}

在得到每个供应商的订货-供货关系后,考虑到各供应商独立,相当于构成了多个单元函数最大值问题。此处考虑到加权平均值的平滑性,可以简单地遍历可选区域内的订货量值变量求解。此过程中的两个约束基本与前一方法相同:只取A,C原材料的供应商;订货量可选范围为$\lbrack 0, \operatorname{min}(6000,\text{ 各供应商历史最大供应量})\rbrack$。

基于此最终选择出4家供应商:[374 307 395 201]。

\subsection{基于概率优化模型的订购方案制定}
本节订购方案将根据上述选出的4家供应商进行规划求解,考虑到供应量的不确定性,尝试在优化中加入概率的元素。为了符号的简单,将上述4家供应商编号[374 307 395 201]重编号为[1 2 3 4],其中第1家供应C类原材料,其余3家供应A类原材料。下面列出本节需要的符号:
\begin{table}[h]
		\centering
		\begin{tabular}{ccc}
			\hline
			符号 &  说明  & 单位 \\
			\hline
      $x_{ij}$ & 第j周对第i家供应商的订购量  & $m^3$ \\
      $y_{ij}$ & 第i家供应商商第j周的供应量 & $m^3$ \\
      $z_{j}$ & 第j周订购原材料的生产量 & $m^3$ \\
      $\epsilon_i$ & 第i家供应商与订购量的误差 & $m^3$ \\
      $\sigma_{\epsilon i}$ & $\epsilon_i$的方差 &  \\
      $store_{j}$ & 第j周的库存所能生产产品的总量 & $m^3$ \\
      $\mu_j$ & $store_{j}$的期望 & $m^3$ \\
      $\sigma_j$ & $store_{j}$的方差 & \\
		\end{tabular}
	\end{table}
  
 有了上述思想和符号准备后,现在开始建立模型。
 
 假设:
 
 1. 假设初始库存恰好为2周生产需求,即$store_0=56400$。
 
 2.每周固定生产28200$m^3$产品。
 
 3.供应量在订购量附近波动,误差$\epsilon_i$服从均值为0的高斯分布,且方差保持时不变,假定标准差均为100,这个数值将在后续稳健性分析中进行更深入研究。
  
  规划目标:最小化采购成本,即
  $$\sum_{j=1}^{24}x_{1j} + 1.2 \sum_{j=1}^{24}\sum_{i=2}^{4}x_{ij}$$
  
  约束条件:
  
  1.订购量非负,即$x_{ij}\ge 0$
  
  2. 由假设3,供应量与订购量的关系为
  $$y_{ij}=x_{ij}+\epsilon_i \sim N(x_{ij},\sigma_{\epsilon i}^2)$$
  
  3. 本周库存等于上一周库存加上本周供应量减去固定的生产数量(假设2),即
  $$store_j=store_{j-1} + \frac{1}{0.72}y_{1j} + \frac{1}{0.6}\sum_{i=2}^{4}y_{ij} -28200$$
  
  4. 库存满足2周生产需求的概率大于给定数值,由假设1,只需
  $$\prob{store_j \ge 56400} \ge 0.945 $$
  
  经过简单计算,可以将上述约束改写为MATLAB cvx可以识别的数学形式:
  \begin{align*}
    	\textbf{min  } &  \sum_{j=1}^{24}x_{1j} +  1.2 \sum_{j=1}^{24}\sum_{i=2}^{4}x_{ij} \\
    	\textbf{subject to  }& \forall i=1,\cdots,4 , j=1,\cdots,24\\
      & x_{ij}\ge 0\\
    	& z_j=\frac{1}{0.72}x_{1j} + \frac{1}{0.6}\sum_{i=2}^{4}x_{ij} \\
      & \mu_j=\mu_{j-1}+z_j-28200\\
      & \sigma_j=j\left(  \frac{\sigma_{\epsilon 1}}{0.6}+\sum_{i=2}^{4} \frac{\sigma_{\epsilon i}}{0.72}\right) \\
      & \mu_j -56400 \ge 1.6\sigma_j
    \end{align*}
    
    利用cvx求解上述优化问题得到最优解,代码和结果详见附录。


\subsection{转运商损耗率的时间序列模型}
由于历史数据没有给出每一周每家转运商的运输量,难以分析供应对损耗的影响,因而作出假设1简化分析。根据此假设,我们可以对转运商的损耗率独立地建立模型。

首先总体观察题目所给附件2中的数据,除了 0 作为缺失值外,我们注意到一个特殊现象:数据中存在大量整数5,与其他数值较小的多位小数极不匹配。我们认为5不是合理的数据,一并作为缺失值处理。接着我们考察每个转运商在240周中缺失值的数量,得到下表结果:
	
	\begin{table}[h]
		\centering
		\begin{tabular}{cccccccc}
			\hline
			T1 & T2& T3& T4& T5& T6& T7& T8  \\
			\hline
			4   &1 &124 &159 &191 & 35 & 41&  49 \\ 
		\end{tabular}
	\end{table}
  
   因为T1、T2、T6、T7、T8这5个序列缺失值数量较少,因而直接用平均值填充来进行分析,而T3、T4、T5序列的缺失值较多,应在后续的模型建立过程中额外考虑。
  
现在对8个序列绘图并观察,由图\ref{1-8}所示。
	\begin{figure}[htbp]
		\centering
		\includegraphics[scale=0.9]{T1-T8}
		\caption{T1至T8序列图像}\label{1-8}
	\end{figure}
  
  根据图像,8个序列形态各异,因此考虑分别建立模型进行预测。
  
  \subsubsection{T1、T2、T6、T7、T8序列基于ARIMA模型的预测}
  以T1序列为例,观察T1序列图像,由图\ref{T1}所示,可以隐约感受到其作为时间序列的周期性与平稳性特征,因此尝试使用ARIMA模型\cite{shumway2000time}进行建模。
\begin{figure}[htbp]
	\centering
	\subfigure{
		\begin{minipage}[t]{0.45\linewidth}
			\includegraphics[scale=0.4]{T1}
		 \caption{T1序列}\label{T1}
		\end{minipage}
	}
	\subfigure{
		\begin{minipage}[t]{0.45\linewidth}
			\includegraphics[scale=0.4]{T1 pred}
		\caption{T1后24周预测}\label{T1 pred}
		\end{minipage}
	}
  \end{figure}



  
  为此,计算序列的自相关与偏自相关并可以得到,T1的自相关周期振荡,而偏自相关在1之后都接近于0,这提示我们T1序列可能符合模型MA(1)。拟合该模型并得到后24周的预测值,见图\ref{T1 pred}。
 
  

  
  类似地,我们可以建立对T2、T6、T7、T8序列的ARIMA模型并得到预测。
  
  \subsubsection{T3、T4序列基于HMM模型的预测}
  以T3为例,仍然首先观察序列图像,由图\ref{T3}所示。
    \begin{figure}[htbp]
		\centering
		\includegraphics[scale=0.4]{T3}
		\caption{T3序列}\label{T3}
	\end{figure}
  
  
  对于该序列(T4具有类似特征),有2个注意点:
  
  1.序列具有较多缺失值,正因为如此,图像中只有中间部分由折线连接。
  
  2.序列在150周附近时值突然变得较大。
  
  为解释第2点,设想T3转运商可能在第150周附近时某种状态发生了改变,导致其运输损耗率偏高。根据这一设想,我们可以使用隐马尔可夫模型(HMM)\cite{shumway2000time}进行建模,将这种状态作为隐状态,并假设在每一状态下损耗率服从高斯分布。这样,该模型可以较好地解释T3转运商损耗率为何发生突变。不仅如此,HMM作为状态-空间模型的一种,可以应对缺失值情形,由此可见使用HMM建模是较为合理的。
  
  在对历史数据进行拟合后,可以得到其对应的状态,如图\ref{HMM fit}所示。
  
   \begin{figure}[htbp]
	\centering
	\subfigure{
		\begin{minipage}[t]{0.45\linewidth}
			\includegraphics[scale=0.4]{HMM fit}
		  \caption{T3序列状态}\label{HMM fit}
		\end{minipage}%
	}%
	\subfigure{
		\begin{minipage}[t]{0.45\linewidth}
			\includegraphics[scale=0.4]{T3 pred}
		  \caption{T3序列后24周预测}\label{T3 pred}
		\end{minipage}
	}

\end{figure}
  

   

  
   在对历史数据的拟合中,模型能够完美将150周附近的数据识别为状态2,其余数据识别为状态1,符合预期解释。后续24周预测见图\ref{T3 pred}。
   
   对T4序列的建模与预测是完全类似的,这里将不作展示。
  

  
  \subsubsection{T5序列分析}
 这里将T5序列单独列出,是由于其特征过于独特。首先,该序列缺失值数量达到191,缺失占比接近80\%;其次,观察其图像(\ref{T5}),难以识别具体的模式,具有较强的随机性。这样,很难找到合适的时间序列模型来拟合它,因此我们将其看做是一个随机分布来分析。
 \begin{figure}[htbp]
	\centering
	\subfigure{
		\begin{minipage}[t]{0.45\linewidth}
			\includegraphics[scale=0.4]{T5}
		\caption{T5序列}\label{T5}
		\end{minipage}%
	}%
	\subfigure{
		\begin{minipage}[t]{0.45\linewidth}
			\includegraphics[scale=0.4]{T5 exp}
		\caption{T5序列核密度估计与指数分布逼近}\label{T5 exp}
		\end{minipage}
	}
	\centering
\end{figure}


  
  现研究去除缺失值后的核密度估计,见图\ref{T5 exp}黑线。因数值一定大于0,考虑用指数分布来逼近,容易计算参数$\lambda$的极大似然估计$\lambda = 0.701448$,逼近效果如图\ref{T5 exp}红线所示。后24周预测则简单使用随机数生成器生成。
  


\subsection{转运方案的0-1规划问题求解}

根据上述决策过程,已经确定了:未来24周的订购方案,转运商的转运损耗率。在上述前提下,需要确定供应商和转运商的指派关系。考虑该指派矩阵,为0-1矩阵,因此可以将该优化问题视作0-1规划问题求解。

\begin{table}[h]
		\centering
		\begin{tabular}{ccc}
			\hline
			符号 &  说明  & 单位 \\
			\hline
      $a_{kj}$ & 第k家转运商第j周的损耗率  & \% \\
      $S_{ij}$ & 第i家供应商商第j周的供应量 & $m^3$ \\
      $T_{ijk}$ & 第i家供应商商第j周与第k家转运商之间的指派关系 & - \\
		\end{tabular}
	\end{table}

定义了上述符号后,考虑该0-1规划问题的约束条件:

\begin{enumerate}

\item 指派矩阵为0-1矩阵。该条件等价于$T_{ijk} \in \{0,1\} $.

\item 每一家供应商所供应的原材料由一家转运商转运。 该条件等价于:$ \sum_{k} T_{ijk} = 1$.

\item 每一家转运商的转运量不超过其最大转运限制。该条件等价于:$\sum_{i} T{ijk} \le 6000 $.

\end{enumerate}

规划的优化目标为:最小化损耗量, 因此总体的0-1规划为:

\begin{align*} 
\textbf{min } & \sum_{ijk} S_{ij} \times  T_{ijk} \times a_{kj} \\
\textbf{subject to } &  \sum_{k} T_{ijk} = 1 \\
& \sum_{i} T_{ijk} \times S_{ij} \le 6000 \\
& T_{ijk} \in \{0,1\} \\
\end{align*}

对于上述0-1规划问题,采用LINGO软件包进行求解,求解部分的代码详见附录。
显然该问题的全局最优解存在,且LINGO确实找到了全局最优解。

\subsection{问题二的效果分析与蒙特卡罗模拟检验}

对于问题2,我们分别得到了两方面的实施方案:对于4家供应商每星期的订购量;将这4家供应商的供应原材料分配至各转运商的具体安排。可以认为,整个建模和求解的过程基本合理,4家供应商选择合理、订购量在合理范围;对于各转运商损耗率的预测合理,在其上得到了满足需求而综合损耗率最低的方案。

此处将进一步地考虑相关的定量分析。此问题中的两个方案均是在考虑了外部随机因素的情况下得到的,例如供应商具体供应量的波动、根据转运商的损耗率的可能变化进行选择。由于随机因素的存在,可能难以确定性的分析给出的方案的具体性质。因此,我们选择蒙特卡罗模拟的方法来检验方案的实施效果。根据本题题目的需要,我们进行两次相对独立的蒙特卡罗模拟:首先是不考虑运输损耗量,研究供应量的波动对于整体生产过程的影响;第二部分是固定各月供应量,计算按照转运方案在转运过程中的损失率。

\subsubsection{供应量波动对生产过程的影响}
此处延续"基于概率优化模型的订购方案制定''一节中的相关假设,并代入对应的求解结果。认为各周、各公司供货量呈独立的正态分布,供货量均值是求解结果中对应量,标准差同样认定为定值100(将在后续内容中进一步讨论)。此子问题中不需要考虑储存、运输成本。模拟程序首先初始化变量$cost=0,store=56400$,对应初始0成本、两周库存。此处“库存”实际上是A,B,C原材料按照各自比例生产产品的量,因为不需要考虑储存从而不需要考虑A,B,C分别的储存量。此后对于每一周进行模拟:

\begin{enumerate}
\item 从库存中扣除每周产量28200
\item 按照4家供应商的供货情况取正态分布随机数,出现小于0或大于6000则置为各自边界量;成本按照比例加入$cost$。
\item 按A,B,C材料比例增加$store$库存变量
\item 若出现$store$低于56400,则报错退出。
\end{enumerate}

我们对于此程序进行了一万次循环,其中没有出现任何一次报错退出;即完全满足了库存量相关约束。同时,统计了24周结束后的总成本和总储存量,如图\ref{MC2}。考虑到每周的产量固定,显然总成本与总储存量线性相关。总成本均值为$5.0497 \times 10^5$,方差为 $ 1.1252\times 10^3 $ ,相对稳定。综上,可以认方案有实施效果,且极具稳定性。

\begin{figure}[h]
\includegraphics[width=\textwidth]{MC2.eps}
\caption{蒙特卡罗模拟得到数据}\label{MC2}
\end{figure}

\subsubsection{转运方案综合损失率的计算}
类似地,建立在求解过程中各种假设的基础上,我们进行一万次了对各转运商、各周损耗率的波动的模拟。

在波动模拟过程中,考虑到ARIMA模型的预测是稳定且收敛的,为节省计算,使用该模型的5条序列损耗率不再模拟。而T3、T4的HMM模型及T5的指数随机分布,因为随机性较强,可以进行模拟。对于T5,只需随机生成随机数;对于T3和T4,根据初始状态和转移概率矩阵,则可以不断生成不同的Markov链,从而完成模拟。

而认定供应量完全按照订购方案的理想情况供应的基础上,我们计算出了4家供应商24周由转运损耗导致的损耗量的统计数值(详见附录)。相对计划供应损失从-5到-75不等,各自有一定波动;每个供应商的平均损耗量为
$$ \mathrm{\mu} = \lbrack -12.4215,  -20.5948,  -20.3157, -19.2731\rbrack $$

对于目前的结果,可以认为总损耗量相比原供应量较小,可以接受;但损耗量的总数是不可忽略的,在问题三中,将考虑相关问题,进行优化。

\section{七、基于稀疏性模型的多目标凸概率优化(问题三)}

\subsection{端到端的规划问题建模}

在问题二中,建立了多步骤的企业规划模型,在问题三中我们尝试将其推广为端到端的规划模型。也即从问题二的先确定订购方案的决策变量,再确定转运方案的决策变量,改为同时确定订购方案和转运方案的决策变量,端到端的规划模型,由于决策变量的自由度更高,很有可能找到更优的解,但相应的求解难度也增大了。未了降低求解难度,在端到端的规划问题中去掉了概率规划产生的不确定性度量,也即假设供应商的满足率恒为1,也即$S=O$。

定义如下的符号:
\begin{table}[h]
		\centering
		\begin{tabular}{ccc}
			\hline
			符号 &  说明  & 单位 \\
			\hline
      $a_{kj}$ & 第k家转运商第j周的损耗率  & \% \\
      $S_{ij}$ & 第i家供应商第j周的供应量 & $m^3$ \\
      $R_{ij}$ & 第i家供应商第j周供应的原材料经过损耗后的剩余量 & $m^3$ \\
      $T_{ijk}$ & 第i家供应商商第j周与第k家转运商之间的指派关系 & - \\
      $d_{ij} $ & 第j周企业储存的第i家供应商的原材料等效的产能 & $m^3$ \\ 
      $b_{ij} $ & 第j周企业储存的第i家供应商的损耗的产能 & $m^3$ \\ 
      $ w_{i}$ & 每家供应商对应的原材料的采购单价比& - \\
      $ u_{i}$ & 每家供应商对应的原材料所等效的产能比 & - \\
     	$ StoreCost$ & 总储存成本 & 元 \\
      $TransportCost$ & 总运输成本 & 元 \\
      $ OrderCost$ & 总订购成本  & 元  \\
      $\alpha, \beta, \gamma$  & 权重 & 元 \\
 		\end{tabular}
	\end{table} 

最终的优化目标为最小化企业的成本,共包括原材料的订购、转运、储存成本。即最小化:
$$ \alpha OrderCost + \beta TransportCost + \gamma StoreCost $$

由于题目中没有给定订购、转运、储存成本之间的比例,可以设置$\alpha=\beta=\gamma=1$。

根据约束条件和优化目标,有:

\begin{align*}
\textbf{minimize  } &\alpha OrderCost + \beta TransportCost + \gamma StoreCost \\
\textbf{subject to }  & OrderCost =  \sum_{ij} w_{i} S_{ij} \\
& TransportCost = \sum_{ij} S_{ij} \\
& StoreCost = \sum_{ij} \frac{d_{ij}}{u_i}  \\
& R_{ij} =  \sum_{k} S_{ij} \times  T_{ijk} \times (1-a_{kj}) \\
&  \sum_{k} T_{ijk} = 1 , T_{ijk} \in \{0,1\} \\
& \sum_{i} T_{ijk} \times S_{ij} \le 6000 \\
& \sum_i d_{ij} \ge 56400 \\
& d_{ij} = d_{i,j-1} + u_{i} R_{ij} - b_{ij} ,j > 1 \\
& \sum_{i} b_{ij} = 28200 \\ 
\end{align*}

上述规划问题为混合整数二阶问题(MIQP),且决策变量过多,比较难找到在限定时间内求解的求解器,单仍然不失为一个参考的基准模型。下面基于假设1的独立性假设,拆分企业的决策规划过程,在问题二的基础上建立改进版的多步骤规划模型。

\subsection{基于范数惩罚和改进概率规划的多步骤规划}
这一节,我们仍沿用问题二的流程与方法,即先求解订购方案的规划问题,再据此方案进行转运商的0-1规划。但在问题二的基础上,针对供货量这个不确定因素,我们据402家企业的情况建立统计模型,对原有的方法进行改进。而对于该多目标规划问题,我们采用1范数惩罚的方式,改进原本的优化目标。

\subsubsection{订购方案的制定}
总体求解思想仍与问题二相同,即在优化模型中加入概率元素。对于规划目标,题目要求尽可能多地订购A而尽可能少地订购B。对于企业而言,成本必然是首要优化的目标,因此考虑在采购成本的基础上加上对B和C类原材料采购的惩罚项,即类似于lasso的方法。就惩罚项的形式而言,我们使用每周B/C类订购总量组成的24维向量的1范数,因为1范数更倾向于导致某些系数为0,即不做采购。

对于约束条件,首先放开4家供应商的限制,将采购范围扩大到全部402家。此外,这402家供应商的地位显然是不同的,为了在优化模型中体现这一点,我们需要对每家供应商的采购量进行限制,并在保证模型简单性的同时对相应的供应量作出更好的估计。基于此,沿用问题二制定订购方案时的符号,模型需要符合如下条件:

\begin{enumerate}
	\item 误差项$\epsilon_i$仍服从高斯分布。
	\item 使用每家供应商历史订购量的0.95分位数作为上界,以保证订购量较少的那些供应商不会喧宾夺主。
	\item 根据历史数据对供应商进行分类,供应量较少的供应商为一类,供应量较多的为另一类。对于前一类供应商,将其供应量求和后视作单独一个供应商,并计算相应的方差。
	\item 对于供货量的估计,使用$y_i=x_i+shift_i+\epsilon_i$作为估计,其中$shift_i$是供应量对订购量的偏移,其值由历史数据中供应与订购残差的平均值来计算得到;$\epsilon_i$的方差$\sigma_{\epsilon i}$由残差的方差来计算得到。
\end{enumerate}

假定问题二中的优化目标为$Obj$,加入范数惩罚后的优化目标为$Obj' = Obj + \lambda \Vert O_C \Vert_1 + \mu \Vert O_B \Vert_1(\lambda> \mu)$.其中$O_B$ 和$O_C$分别表示两类原材料的订购量矩阵,在该问题的求解中,设$\lambda=0.1,\mu=0.05$,。经过实验验证,该参数的设置对解影响很小,对于该惩罚项参数的稳健性详见模型的稳健性分析一节。

由于优化问题与问题二极为相似,这里不再列出,详见附录代码。本节数据的预处理使用了统计软件R,而优化问题则利用MATLB的cvx工具箱成功求解。

\subsubsection{转运方案的制定}

给定订购方案之后,求解与问题二中相同的0-1规划即可得到转运方案。但考虑到求解402家企业的转运方案的全局最优解的复杂度较高,考虑在LINGO求解器中加入时间限制,求解器给出的0-1规划的下界为129188.9$m^3$,而当前找到的最优的可行解为129191.1 $m^3$, 2$m^2$的误差实际上对结果并没有过大的影响,该结果距离求解器给出的最优下界仅有0.0015\%的误差,在精度要求内可以忽略不计,因此可以认为求解器找到了最优解。

\subsection{问题三的方案效果分析与蒙特卡罗模拟检验}
类似于问题2,本节中对于问题3也同样采用蒙特卡罗方法进行检验;但此处由于同时考虑了订购与转运两方面的内容,将需要在同一次模拟中模拟两方面内容的随机性。基于前文的假设,此处利用建模时对于供货商的分类、关于供货量变化的假设,以及问题二中已得到的10000条损耗率波动模拟数据。共进行10000次试验,依次取损耗率波动数据中的一条,并生成供货量数据。对于这两方面的处理基本与问题二对应部分相同。但有所不同的时此处分别考虑了A、B、C原料的储存和运输,仅在生产时、计算采购成本时用对应系数修正。另外,还需要考虑有多种材料库存时,该用哪一种材料的问题;实际上由于储存成本的性质,只需要按照生产等量产品消耗量顺序,依此使用C、B、A即可。

实际程序编写中,供货量的生成过程中以订购量-$shift$为均值,解题过程中统计标准差作为标准差。而对于解题中进行的小供应商合并为大供应商的情况,此处将统计的$n$个小供应商的合计标准差$\sigma$平均分拆给各小供应商,标准差变为$\frac{\sigma}{\sqrt{n}}$。考虑到小供应商的供应量很少,这样的做法合理。

模拟的具体流程为:
\begin{enumerate}
\item 每次外层循环中,提前生成24周的供应量、载入损耗率模拟数据。库存数量设为56400*0.6单位的A材料,即恰能满足库存约束。
\item 计入库存费用;
\item 从库存中扣除每周产量28200;库存按照C、B、A顺序依此扣去;
\item 供应量的成本按照比例加入$cost$。;
\item 按A,B,C材料比例增加$store$库存变量;
\item 若出现$store$低于56400,则报错退出。
\end{enumerate}

本次试验共进行10000次; 其中出现了536次未满足库存条件而退出的情况,即方法的可靠率为94.54\%。总成本的均值为$5.2179\times 10^6$,方差为$3.0547\times 10^5$。下图\ref{MC3}分别展示了24周生成生产结束后的总成本与各库存结余情况,以及总成本的分布。生产中主要订购A,与总成本有着显著的线性关系,但有少部分的B、C分布。基于此结果,可以认为我们得到的方法比较稳健,且即使考虑到了供货量波动和随机转运损耗,仍可以有一定的库存结余。

\begin{figure}[htbp]
\includegraphics[width=\textwidth]{MC3.eps}
\caption{问题三整体生产过程的蒙特卡罗模拟} \label{MC3}
\end{figure}

\section{八、基于供应商潜力估计的二分搜索算法(问题四)}

\subsection{基于二分法的求解方法}
利用二分法的思想,可以将问题四转化为问题三。根据转运损耗率与供应量无关的假设,可以将订购方案的决策和转运方案的决策视作两个独立的决策过程,仍然考虑多阶段规划模型。问题四要求寻找企业可以提升的最大产能,本质上可以转化为一个存在性问题,等价于寻找使得问题二和问题三中的订购规划问题存在可行解的最大产能。显然,该产能满足二分法中的单调性条件,因此只要给定二分法的上下限,便可以使用问题二或问题三中的规划模型作为二分法的存在性判断的子程序。

二分法的下限为当前的产能$M=28200$,而其上限为一先验条件。简单地,假设最大的产能可以提升为当前产能的$N$倍,则二分法的上限为$MN$. 设规划模型的复杂度为$T$,则该二分法只需要在$\log{N} \times T$的时间内就可以完成。

\subsection{基于供应商统计模型的供应商潜力估计}

供应商的供应量,为企业产能提高的限制之一,因此对于供应商供应能力的估计在该问题中较为关键。上述基于二分的方法,在求解最优的订购方案的时候,沿用了问题三中给出的供应商供应量的概率模型:本质上是将其5年内的供应量的经验分布函数作为其概率分布的估计。由于经验分布函数依概率收敛于真实分布函数,因此上述方法有其合理性。但使用经验分布函数将导致推断出的供应商最大的可能的供应量不可能超过历史的最大供应量,在某种程度上是对供应商的供应潜力的一种限制。因此在本节中,考虑重新为供应商的供应量建立统计模型。

改进后的统计模型建立在问题三的模型的基础上,也即供应商已经按照其供货特征被分为几类,下面分别对其建立不同的统计模型:

\begin{enumerate}

\item 对于极其不重要的供应商,供应量矩阵具有很大的稀疏性(零元很多),由于供应量为非负数且均为离散值,符合泊松分布的特性。因此,将这部分供应商用泊松分布建模。

\item 对于其他的供应商,假设其供应量服从高斯分布。

\end{enumerate}

采用极大似然估计可以得到分布函数中的参数,从而给出所有402家供应商的统计模型。该改进手段的优点是使得最大供应量可以突破历史最大供应量的限制,从而使得对供应商的潜力的估计更为合理。

给定置信度$\alpha$, 选取每家供应商的分布函数的上$1-\alpha$分位数作为其在该置信度下的供应潜力估计。将新的限制条件代入问题三的概率优化模型中求解,同样使用二分法寻找最大产能,可以得到若供应商的潜力发生改变,在给定置信度下的最大产能。调用二分方法可以发现,当企业的产能为7.64万$m^3$的时候不存在可行解,但当产能为7.65万$m^3$的时候规划模型可以找到相应的解,因此二分法给出的最大产能为7.64万$m^3$. 

\subsection{基于线性规划的转运方案制定}

问题二和问题三中转运方案的制定为0-1规划模型,但在问题四中,由于企业的产能得到提升,问题二和问题三中假设每一家供应商所供应的原材料仅由一家转运商转运的条件不应再成立。根据上面的订购方案的求解结果,存在着某些供应商的单周供应量超过每家转运商的6000$m^3$的最大转运容量限制,需要由多家转运商进行转运。因此在问题四的转运方案的制定过程中,应该去除该限制,该改动使得原本的0-1规划问题变为更易于求解的线性规划问题。

在问题二和问题三所定义的符号的基础上,将原本的指派矩阵$T$重新定义为转运商的决策矩阵,也即矩阵元素$T_{ijk}$表示第i家供应商在第j周由第k家转运商所承担的转运量。

根据上述分析,问题四所需要求解的线性规划为:

\begin{align*} 
\textbf{minimize } & T_{ijk} \times a_{kj} \\
\textbf{subject to } &  \sum_{k} T_{ijk} = S_{ij} \\
& \sum_{i} T_{ijk} \le 6000 \\
\end{align*}

使用LINGO求解,代码与结果详见附录和支撑材料。

%----------- 六、模型的分析与检验 ----------
\section{九、模型的分析}

\subsection{稳健性分析}

在上述建立的数学模型中,对于问题中没有给定的数值或者超参数,我们经过分析进行相对合理地赋值。但这部分超参数会影响模型的表现能力,但在这一节中我们验证了我们所建立的数学模型对这些超参数是不敏感的,因此我们所建立的模型是鲁棒的。

\subsubsection{问题二订购方案的稳健性}
在问题二订购方案的优化模型中,标准差被设定为100,这里我们将变化这个值进而考察最优值并计算平均每周最优解与基准模型的frobenius范数来衡量最优解的差距。计算结果如下表所示,可以看到最优值(采购成本)随着标准差的增加而增加。这是符合预期的,因为标准差越大,为了保证库存达到需求的概率,就必须增加订购量。而对于最优解的差距,考虑到每周约1.7万的订购量,即便是1000情形的600,相对波动约为3.5\%,并不是很大,表明模型相对稳健。

\begin{table}[h]
    \centering
\begin{tabular}{cccc}
	
	序号&标准差&最优值& 范数 \\
	\hline\hline
	1&10&489062& 60.12\\
	\hline
	2&100&504960& 0\\
	\hline
	3&200&522624& 66.81\\
	\hline
	4&500&575616& 267.23\\
	\hline
	5&1000&663936& 601.28\\
	\hline
\end{tabular}
\end{table}

\subsubsection{问题三订购方案的稳健性}
相对于问题二,问题三的优化模型中加入了对订购B、C类材料的的惩罚项,这里将变化惩罚项前的系数来分析模型的稳健性。同样考察最优值和最优解与基准模型的差距,结果如下表所示。可以看到最优值几乎不改变,最优解差距仅为个位数,表明模型具有极强的稳健性。
\begin{table}[h]
    \centering
\begin{tabular}{cccc}
	
	序号&系数&最优值& 范数 \\
	\hline\hline
	1&0.1,0.05&600526& 0\\
	\hline
	2&0.05,0.1&600527& 10.22\\
	\hline
	3&0.5,0.1&600526& 3.18\\
	\hline
	4&1,0.5&600526& 8.30\\
	\hline
	5&10,1&600526& 6.95\\
	\hline
\end{tabular}
\end{table}

\subsection{误差分析}

\begin{enumerate}

\item 对于订购方案的决策,将问题转化为凸优化问题,找到的解即为全局最优解,所得解与理论最优解的误差仅仅来自于取决于迭代方法的停止准则产生的数值误差。

\item 对于0-1规划问题,问题二中制订的转运方案找到了全局最优解,而问题三中给出的解与算法给出的下界之间仅有0.0015\%的最大误差,可以忽略不计。

\item 尽管建立的订购量-供应量的相关模型和转运商的损耗率模型(如HMM模型)等存在随机变量产生的误差,但经过蒙特卡罗方法模拟可见上述误差很小,对解的可行性几乎没有影响。

\end{enumerate}

\iffalse
模型的分析与检验的内容也可以放到模型的建立与求解部分,这里我们单独抽出来进行讲解,因为这部分往往是论文的加分项,很多优秀论文也会单独抽出一节来对这个内容进行讨论。
模型的分析 :在建模比赛中模型分析主要有两种,一个是灵敏度(性)分析,另一个是误差分析。灵敏度分析是研究与分析一个系统(或模型)的状态或输出变化对系统参数或周围条件变化的敏感程度的方法。其通用的步骤是:控制其他参数不变的情况下,改变模型中某个重要参数的值,然后观察模型的结果的变化情况。误差分析是指分析模型中的误差来源,或者估算模型中存在的误差,一般用于预测问题或者数值计算类问题。
模型的检验:模型检验可以分为两种,一种是使用模型之前应该进行的检验,例如层次分析法中一致性检验,灰色预测中的准指数规律的检验,这部分内容应该放在模型的建立部分;另一种是使用了模型后对模型的结果进行检验,数模中最常见的是稳定性检验,实际上这里的稳定性检验和前面的灵敏度分析非常类似。
\fi

\section{十、模型的评价与推广}

\iffalse
注:本部分的标题需要根据你的内容进行调整,例如:如果你没有写模型推广的话,就直接把标题写成模型的评价与改进。很多论文也把这部分的内容直接统称为“模型评价”部分,也是可以的。
\fi 

\subsection{模型的优点}

\begin{enumerate}
\item 使用集成学习模型衡量企业重要性,有效降低了模型的偏差。
\item 分别使用ARIMA、HMM、指数分布模型等具有针对性的模型预测不同特点的转运商的损耗率。
\item 所建立的模型对于超参数的设置不敏感,详见稳健性分析一节。
\item 创新型地建立了移动加权平均模型用于供应量的回归,本质上类似于高斯去噪过程\cite{gonzale2002digital},有效地降低了噪声数据的影响。
\item 采用概率优化模型,定量地衡量优化中的不确定性因素。
\item 使用蒙特卡罗方法,利用统计模拟实验有力地说明结果的有效性。

\end{enumerate}
\subsection{模型的缺点}
\begin{enumerate}
	\item 在优化模型中,为了适应概率的复杂度,在估计供应量方面简单地使用一些线性的方法,并且误差项均假设为同方差的高斯噪声,这与实际不尽相符。
	\item ARIMA模型的预测一般会收敛到一个值,因此不太适合做长时间的预测。
	\item 0-1规划为指数型复杂度,不适用于推广到供应商数目过多的情况,在该情况下应该优先使用基于剪枝和贪心的算法寻找候选供应商。
	
\end{enumerate}

\subsection{模型的推广}
\begin{enumerate}

\item 本文所建立的模型针对的问题是极具代表性的,可以推广到各类企业的决策过程中。
\item 模型对于不确定因素的量化,对于随机环境下的企业决策具有参考意义。
\item 对供应商、转运商建立的统计模型可以运用到其他相似的具有时序特征、相关关系、稀疏性特点的数据中。
\end{enumerate}

%----------- 参考文献 ----------
\bibliographystyle{unsrt} %规定了参考文献的格式
\begin{center}
\bibliography{reference} %调出LaTeX生成参考文献列表
\end{center}

\iffalse
\textcolor{red}{(所有引用他人或公开资料(包括网上资料)的成果必须按照科技论文的规范列出参考文献,并在正文引用处予以标注。}
\textcolor{red}{常见的三种参考文献的表达方式(标准不唯一):
书籍的表述方式为: [编号] 作者,书名,出版地:出版社,出版年月。
期刊杂志论文的表述方式为: [编号] 作者,论文名,杂志名,卷期号:起止页码,出版年。
网上资源(例如数据库、政府报告)的表述方式为: [编号] 作者,资源标题,网址,访问时间。)}
\fi 

%----------- 附录 ----------
\newpage
\section{附录}

\iffalse
\begin{table}[htbp]
    \centering
    \begin{tabular}{|p{14.0cm}|}
 %指定单元格宽度, 并且水平居中。
    \hline
    \textbf{附录1} \\ %换行 
    \hline
    介绍:支撑材料的文件列表  \\ 
    \\
    \\
    \\
    \hline
    \end{tabular}
\end{table}

\begin{table}[htbp]
    \centering
    \begin{tabular}{|p{14.0cm}|}
 %指定单元格宽度, 并且水平居中。
    \hline
    \textbf{附录2} \\ %换行 
    \hline
        \\ 
    \\
    \\
    \\
    \hline
    \end{tabular}
\end{table}
\fi 


\noindent \textbf{附录A:提交结果} \\
\begin{itemize}
\item 附件A 订购方案数据结果.xlsx 
\item 附件B 转运方案数据结果.xlsx 
\end{itemize}

\hspace*{\fill} \\ 
\noindent \textbf{附录B:支撑材料列表} \\

\begin{itemize}

\item data: 文件夹下存放数据文件,包括程序的IO数据和中间结果
\item prepocess: 文件夹下存放数据格式转换,数据预处理相关的文件
\item LINGO: 存放.lg4文件,详见附录C
\item R: 存放.R文件,详见附录C
\item MATLAB: 存放.m程序文件和.mat矩阵数据,详见附录C
\item Python: 存放.py文件,详见附录C
\end{itemize}

\hspace*{\fill} \\ 

\noindent \textbf{附录C:代码} \\

运行环境说明

\begin{itemize}
\item .R文件在R 4.0.3版本上可以运行
\item .m文件在matlab R2020a版本上可以运行
\item .py文件在Python3.8版本上可以运行
\item .lg4文件在LINGO18.0版本上可以运行
\end{itemize}

代码文件说明

\begin{itemize} 

\item p1-cap.py: 可视化历史5年内的原材料储存量曲线。

\begin{lstlisting}[language={Python}]
import pandas as pd 
import numpy as np
import matplotlib.pyplot as plt 

#读取数据并输出

sExcelFile_1 = 'data\data1.xlsx'
sExcelFile_2 = 'data\data2.xlsx'
df_order = pd.read_excel(sExcelFile_1,sheet_name=0,engine='openpyxl')
df_supply = pd.read_excel(sExcelFile_1,sheet_name=1,engine='openpyxl')
df_transport = pd.read_excel(sExcelFile_2,sheet_name=0,engine='openpyxl')

array_order = np.array(df_order)[:-2,:]
# 出现两行NAN值,预先去除
array_supply = np.array(df_supply)
array_transport = np.array(df_transport)

# ABC三类原材料
#print(array_supply[:,1])

print(array_supply.shape) #(402,242)

# 第一列:名称
# 第二列:类别

array_cap = np.zeros(240)

cap = 0
for j in range(2,242):
    for i in range(0,402):
        ty = array_supply[i,1]
        if ty == 'A':
            cap += array_supply[i,j] / 0.6
        elif ty == 'B':
            cap += array_supply[i,j] / 0.66
        elif ty == 'C':
            cap += array_supply[i,j] / 0.72
    cap -= 28200
    array_cap[j-2] = cap
print(array_cap)
plt.plot(array_cap)
plt.show()

cap = 0
for j in range(2,242):
    for i in range(0,402):
        ty = array_order[i,1]
        if ty == 'A':
            cap += array_order[i,j] / 0.6
        elif ty == 'B':
            cap += array_order[i,j] / 0.66
        elif ty == 'C':
            cap += array_order[i,j] / 0.72
    cap -= 28200
    array_cap[j-2] = cap
print(array_cap)
plt.plot(array_cap)
plt.show()
\end{lstlisting}
\item C1-clustering.R:问题一中衡量供应商重要性的层次聚类模型。

\begin{lstlisting}[language=R]
d <- read.table('DS/data/d.txt', header = F)
s <- read.table('DS/data/s.txt', skip = 1, header = F)

sA <- s[s[2]=='A',]
sB <- s[s[2]=='B',]
sC <- s[s[2]=='c',]
sA.w <- sA[,c(-1,-2)]/0.6
sB.w <- sB[,c(-1,-2)]/0.66
sC.w <- sC[,c(-1,-2)]/0.72
s.w <- s
s.w[s[2]=='A',c(-1,-2)] <- sA.w
s.w[s[2]=='B',c(-1,-2)] <- sB.w
s.w[s[2]=='C',c(-1,-2)] <- sC.w

feature <- function(s) {
 s.mean <- apply(s[,c(-1,-2)], 1, mean)
 s.rate <- s[,c(-1,-2)]/d[,c(-1,-2)]
 s.rate[is.nan(as.matrix(s.rate))] <- NA
 s.rate.mean <- apply(s.rate, 1, mean, na.rm = T)
 s.rate.var <- apply(s.rate, 1, var, na.rm = T)
 return(cbind(s.mean, s.rate.mean, s.rate.var))
}
s.w.feature <- feature(s.w)

X <- scale(s.w.feature)
X[1] <- X[1]
hc.complete <- hclust(dist(X), method="complete")
hc.average <- hclust(dist(X), method="average")
opar <- par(no.readonly = T)
par(mfrow=c(1,2))
plot(hc.complete, main="Complete Linkage", 
     xlab="", sub="", cex=.9)
plot(hc.average, main="Average Linkage", 
     xlab="", sub="", cex=.9)
par(opar)
class.c <- cutree(hc.complete, 2)
table(class.c)
aggregate(s.w.feature, by=list(cluster=class.c), median)

class.a <- cutree(hc.average, 3)
table(class.a)
aggregate(s.w.feature, by=list(cluster=class.a), mean)

\end{lstlisting}

\item p1-imp.py: 使用集成学习模型对企业重要性进行评价,代码中包括了供应量优先模型、熵权模型、聚类模型的实现,机器学习模型部分使用了sklearn机器学习包。

\begin{lstlisting}[language={Python}]
import pandas as pd 
import numpy as np
import matplotlib.pyplot as plt 
from sklearn.cluster import KMeans
#读取数据并输出

sExcelFile_1 = 'data\data1.xlsx'
sExcelFile_2 = 'data\data2.xlsx'
df_order = pd.read_excel(sExcelFile_1,sheet_name=0,engine='openpyxl')
df_supply = pd.read_excel(sExcelFile_1,sheet_name=1,engine='openpyxl')
df_transport = pd.read_excel(sExcelFile_2,sheet_name=0,engine='openpyxl')

# 出现两行NAN值,预先去除
array_order = np.array(df_order)[:-2,:]
array_supply = np.array(df_supply)
array_transport = np.array(df_transport)

#根据种类进行归一化
def scale_by_type(array):
    for i in range(0,402):
        ty = array[i,1]
        if ty == 'A':
            array[i,2:] = array[i,2:] / 0.6
        elif ty == 'B':
            array[i,2:] = array[i,2:] / 0.66
        elif ty == 'C':
            array[i,2:] = array[i,2:] / 0.72
    return array

array_order = scale_by_type(array_order)
array_supply = scale_by_type(array_supply)

def min_max_scale(a):
    a = (a - np.min(a)) / (np.max(a) - np.min(a))
    return a 

#供应量优先模型:根据总供应平均供应量确定企业重要性
def method_sum_supply():
    # 根据总供应量选择重要的企业
    sum_supply = np.mean(array_supply[:,2:],1)
    sum_supply = min_max_scale(sum_supply)
    sort_score = np.argsort(sum_supply)
    imp = sort_score[-50:] + 1
    return sum_supply, imp

def get_features():

    # 特征提取
    rate = array_supply[:,2:] / (array_order[:,2:] + 1e-12)
    rate[rate>1e8] = None
    
    mrate = np.mean(rate,1)
    # 用方差衡量稳定性
    var = 1 - np.var(rate,1)
    msupply = np.mean(array_supply[:,2:],1)

    print(msupply.shape)

    # 可以选择做聚类... 

    mrate = min_max_scale(mrate)
    var = min_max_scale(var)
    msupply = min_max_scale(msupply)

    #生成特征矩阵

    features = np.zeros((402,3))
    features[:,0] = msupply
    features[:,1] = mrate
    features[:,2] = var 

    return features

#基于熵确定权重
def method_end():
    # 特征提取

    features = get_features()
    #print(features)
    k=1/np.log(3)
    yij=features.sum(axis=0)
    pij=features/yij

    #计算pij
    test=pij*np.log(pij + 1e-5)
    test=np.nan_to_num(test)
    ej=-k*(test.sum(axis=0))
    #计算每种指标的信息熵
    wi=(1-ej)/np.sum(1-ej)
    #计算每种指标的权重

    #wi = np.array([1,0,0])
    scores = np.sum(features * wi,1)
    #print(scores)
    sort_score = np.argsort(scores)
    imp = sort_score[-50:] + 1
    
    return scores, imp

def method_cluster():
    features = get_features()
    kmeans = KMeans(n_clusters=3)
    kmeans.fit(features)
    y_kmeans = kmeans.predict(features)

    index = np.argwhere(y_kmeans==y_kmeans[229-1])
    index = index.squeeze() + 1
    print(index)
    return index 

def intersect(a,b):
    ans = 0
    for aa in a:
        for bb in b:
            if aa == bb:
                ans += 1
                break
    return ans 

if __name__ == '__main__':
    sc1,imp1 = method_sum_supply()
    sc2,imp2 = method_end()

    sc = sc1 + sc2 
    imp = np.argsort(sc)[-50:] + 1
    print(imp)
    
    #print(intersect(imp1,imp2))

    # 比较两种方法的的得分
    # 相等的共有41家企业

    #imp3 = method_cluster()
    #print(imp3)
    #print('number of imp3',len(imp3))
    #print(intersect(imp2,imp3))

		\end{lstlisting}

\item p2-num.py: 问题二中最少供应商数目的求解,基于剪枝过滤后贪心算法的代码实现。

\begin{lstlisting}[language={Python}]
import pandas as pd 
import numpy as np
import matplotlib.pyplot as plt 
from sklearn.cluster import KMeans
from sklearn.mixture import GaussianMixture as GMM

#读取数据并输出

sExcelFile_1 = 'data\data1.xlsx'
sExcelFile_2 = 'data\data2.xlsx'
df_order = pd.read_excel(sExcelFile_1,sheet_name=0,engine='openpyxl')
df_supply = pd.read_excel(sExcelFile_1,sheet_name=1,engine='openpyxl')
df_transport = pd.read_excel(sExcelFile_2,sheet_name=0,engine='openpyxl')

# 出现两行NAN值,预先去除
array_order = np.array(df_order)[:-2,:]
array_supply = np.array(df_supply)
array_transport = np.array(df_transport)

#根据种类进行归一化

def scale_by_type(array):
    for i in range(0,402):
        ty = array[i,1]
        if ty == 'A':
            array[i,2:] = array[i,2:] / 0.6
        elif ty == 'B':
            array[i,2:] = array[i,2:] / 0.66
        elif ty == 'C':
            array[i,2:] = array[i,2:] / 0.72
    return array

#进行等效化
array_order = scale_by_type(array_order)
array_supply = scale_by_type(array_supply)

def min_max_scale(a):
    a = (a - np.min(a)) / (np.max(a) - np.min(a))
    return a 

#scale过后就不需要考虑type信息,先将type信息记录在相应的数组中
array_type = array_supply[:,1]
#print('type',array_type)
array_order = array_order[:,2:]
array_supply = array_supply[:,2:]

def greedy_search(sort_max_supply,max_valid_supply):
    i = 401
    s = 0
    selected_company = []
    while s < 28200:
        index = sort_max_supply[i]
        s += max_valid_supply[index]
        i -=1 
        selected_company.append(index+1)

    #print(selected_company)
    return selected_company

def greedy_method():
    # 数据稀疏性分析
    #print(np.sum(array_supply[:,2:]>=0))
    #print(np.sum(array_supply[:,2:]>0))

    mean_valid_supply = np.zeros(402)
    for i in range(0,402):
        s = 0 
        cnt = 0
        for j in range(0,240):
            if array_supply[i,j]>0:
                s += array_supply[i,j]
                cnt += 1
        #print(cnt)
        mean_valid_supply[i] = s / cnt 

    # 尝试使用贪心算法求解,使用有效的均值和最大值分别衡量
    
    #print(np.sort(mean_valid_supply))
    #print(np.sort(max_supply))

    max_valid_supply = np.max(array_supply,1)
    maxA = 6000 / 0.6
    maxB = 6000 / 0.66
    maxC = 6000 / 0.72 

    #最大供应量需要受到限制
    for i in range(402):
        ty = array_type[i]
        if ty == 'A':
            if max_valid_supply[i] > maxA:
                max_valid_supply[i] = maxA
        elif ty == 'B':
            if max_valid_supply[i] > maxB:
                max_valid_supply[i] = maxB
        elif ty == 'C':
            if max_valid_supply[i] > maxC:
                max_valid_supply[i] = maxC
    #print(max_valid_supply)
    
    sort_mean_supply = np.argsort(mean_valid_supply)
    sort_max_supply = np.argsort(max_valid_supply)

    # 贪心算法对于max_supply求解
    selected_company = greedy_search(sort_max_supply,max_valid_supply)
    print(selected_company)

    # 按照最大的求解: [201, 307, 395]
    # 但没有考虑订购量和供应量的关系

    selected_company = greedy_search(sort_mean_supply,mean_valid_supply)
    print(selected_company)

if __name__ == '__main__':
    # 贪心算法
    greedy_method()
    
\end{lstlisting}
\item zy-prediction.R: 实现问题二中关于转运商损耗率预测模型,数出'zy-pred.csv'。

\begin{lstlisting}[language=R]

zy <- read.table('DS/data/zy.txt', header = T)
zy <- zy[-1]
zy[zy == 5] <- 0
zy[zy == 0] <- NA

# 3,4,5 have many NA
apply(zy, 1, function(x) {
  sum(is.na(x))
})

# use mean to fill NA exclude T3,T4,T5
zy.mean <- apply(zy, 1, mean, na.rm = T)
for (i in c(1, 2, 6, 7, 8)) {
  zyi <- zy[i, ]
  zy[i, is.na(zyi)] <- zy.mean[i]
}

# data frame to vector
T1 <- as.vector(as.matrix(zy[1, ]))
T2 <- as.vector(as.matrix(zy[2, ]))
T3 <- as.vector(as.matrix(zy[3, ]))
T4 <- as.vector(as.matrix(zy[4, ]))
T5 <- as.vector(as.matrix(zy[5, ]))
T6 <- as.vector(as.matrix(zy[6, ]))
T7 <- as.vector(as.matrix(zy[7, ]))
T8 <- as.vector(as.matrix(zy[8, ]))
# plot
par(mfrow = c(8,1))
plot(T1)
plot(T2)
plot(T3)
plot(T4)
plot(T5)
plot(T6)
plot(T7)
plot(T8)
par(mfrow = c(1,1))

## T1
# arima model
library(astsa)
# ma(1)
plot(T1,type = "o")
acf2(T1)
T1.pred <- sarima.for(T1, 24, 0, 0, 1)

## T2
# arima(1,1,0)
plot(T2,type = "o")
acf2(T2)
T2.d <- diff(T2)
acf2(T2.d)
T2.pred <- sarima.for(T2, 24, 1, 1, 0)

## T3
# hmm
plot(T3,type = "o")
library(depmixS4)
hmm.T3 <-
  depmix(T3 ~ 1,
         nstates = 2,
         data = data.frame(T3),
         family = gaussian())
set.seed(574846)
summary(fm <- fit(hmm.T3))

plot(
  T3,
  main = "",
  ylab = 'T3',
  type = 'h',
  col = gray(.7)
)
text(
  T3,
  col = 6 * posterior(fm)[, 1] - 2,
  labels = posterior(fm)[, 1],
  cex = .9
)
# parameters
para = as.vector(getpars(fm))
mtrans = matrix(para[3:6], byrow = TRUE, nrow = 2)
mu = para[c(7, 9)]
sigma = para[c(8, 10)]
generate.sample = function(n, m, mu, sigma, Mtrans, ostate)
{
  # n  length
  # m  # of states
  # Mtrans  transition matrix
  # ostate  origin state
  mvect = 1:m
  state = numeric(n)
  state[1] = ostate
  for (i in 2:n)
    state[i] = sample(mvect , 1, prob = Mtrans[state[i - 1] ,])
  y = rnorm(n, mu[state], sigma[state])
  list(y = y, state = state)
}
set.seed(1234567)
T3.pred <- generate.sample(24, 2, mu, sigma, mtrans, 2)
T3.pred
plot(
  T3.pred$y,
  main = "",
  ylab = 'T3 prediction',
  type = 'h',
  col = gray(.7)
)
text(
  T3.pred$y,
  col = 6 * posterior(fm)[, 1] - 2,
  labels = T3.pred$state,
  cex = .9
)

## T4
# hmm
plot(T4,type = "o")
hmm.T4 <-
  depmix(T4 ~ 1,
         nstates = 2,
         data = data.frame(T4),
         family = gaussian())
set.seed(574846)
summary(fm <- fit(hmm.T4))

plot(
  T4,
  main = "",
  ylab = 'T4',
  type = 'h',
  col = gray(.7)
)
text(
  T4,
  col = 6 * posterior(fm)[, 1] - 2,
  labels = posterior(fm)[, 1],
  cex = .9
)

# parameters
para = as.vector(getpars(fm))
mtrans = matrix(para[3:6], byrow = TRUE, nrow = 2)
mu = para[c(7, 9)]
sigma = para[c(8, 10)]
# prediction
set.seed(684586844)
T4.pred <- generate.sample(24, 2, mu, sigma, mtrans, 2)
T4.pred

## T5
# seems no pattern
plot(T5)
T5.narm <- na.exclude(T5)
plot(density(T5.narm),main = 'density of T5')
# exponential distribution
mu <- mean(T5, na.rm = T)
sigma <- var(T5, na.rm = T)
lam <- 1 / mu
x <- seq(0,5,length.out = 100)
lines(x,dexp(x,rate=lam),col = 2)
# prediction
T5.pred <- rexp(24, lam)

## T6
# ar(1)
plot(T6,type = "o")
acf2(T6)
T6.pred <- sarima.for(T6, 24, 1, 0, 0)

## T7
# arima(1,1,0)
plot(T7,type = "o")
acf2(T7)
T7.d <- diff(T7)
acf2(T7.d)
T7.pred <- sarima.for(T7, 24, 1, 1, 0)

## T8
# arima(1,1,0)
plot(T8,type = "o")
acf2(T8)
T8.d <- diff(T8)
acf2(T8.d)
T8.pred <- sarima.for(T8, 24, 1, 1, 0)

zy.pred <- rbind(
  T1.pred$pred,
  T2.pred$pred,
  T3.pred$y,
  T4.pred$y,
  T5.pred,
  T6.pred$pred,
  T7.pred$pred,
  T8.pred$pred
)
row.names(zy.pred) <- c('T1', 'T2', 'T3', 'T4',
                        'T5', 'T6', 'T7', 'T8')
colnames(zy.pred) <- 1:24

write.csv(zy.pred,
            'zy_pred.csv')

\end{lstlisting}

\item zy-sim.R: 根据对损耗率已建立的模型模拟10000次,输出'zy-simulation.txt'.

\begin{lstlisting}[language=R]
# data
zy <- read.table('DS/data/zy.txt', header = T)
zy <- zy[-1]
zy[zy == 5] <- 0
zy[zy == 0] <- NA

# 3,4,5 have many NA
apply(zy, 1, function(x) {
  sum(is.na(x))
})

# use mean to fill NA exclude T3,T4,T5
zy.mean <- apply(zy, 1, mean, na.rm = T)
for (i in c(1, 2, 6, 7, 8)) {
  zyi <- zy[i,]
  zy[i, is.na(zyi)] <- zy.mean[i]
}

# data frame to vector
T1 <- as.vector(as.matrix(zy[1,]))
T2 <- as.vector(as.matrix(zy[2,]))
T3 <- as.vector(as.matrix(zy[3,]))
T4 <- as.vector(as.matrix(zy[4,]))
T5 <- as.vector(as.matrix(zy[5,]))
T6 <- as.vector(as.matrix(zy[6,]))
T7 <- as.vector(as.matrix(zy[7,]))
T8 <- as.vector(as.matrix(zy[8,]))

# simulation
library(depmixS4)
# T3
hmm.T3 <-
  depmix(T3 ~ 1,
         nstates = 2,
         data = data.frame(T3),
         family = gaussian())
fm <- fit(hmm.T3)
para3 = as.vector(getpars(fm))
mtrans3 = matrix(para3[3:6], byrow = TRUE, nrow = 2)
mu3 = para3[c(7, 9)]
sigma3 = para3[c(8, 10)]

# T4
hmm.T4 <-
  depmix(T4 ~ 1,
         nstates = 2,
         data = data.frame(T4),
         family = gaussian())
fm <- fit(hmm.T4)
para4 = as.vector(getpars(fm))
mtrans4 = matrix(para4[3:6], byrow = TRUE, nrow = 2)
mu4 = para4[c(7, 9)]
sigma4 = para4[c(8, 10)]


generate.sample = function(n, m, mu, sigma, Mtrans, ostate)
{
  # n  length
  # m  # of states
  # Mtrans  transition matrix
  # ostate  origin state
  mvect = 1:m
  state = numeric(n)
  state[1] = ostate
  for (i in 2:n)
    state[i] = sample(mvect , 1, prob = Mtrans[state[i - 1] , ])
  y = rnorm(n, mu[state], sigma[state])
  list(y = y, state = state)
}

mu <- mean(T5, na.rm = T)
lam <- 1 / mu
for (i in 1:10000) {
  T3.pred <- generate.sample(24, 2, mu3, sigma3, mtrans3, 2)
  T4.pred <- generate.sample(24, 2, mu4, sigma4, mtrans4, 2)
  T5.pred <- rexp(24, lam)
  zy.pred <- rbind(
    T1.pred$pred,
    T2.pred$pred,
    abs(T3.pred$y),
    abs(T4.pred$y),
    T5.pred,
    T6.pred$pred,
    T7.pred$pred,
    T8.pred$pred
  )
  row.names(zy.pred) <- c('T1', 'T2', 'T3', 'T4',
                          'T5', 'T6', 'T7', 'T8')
  colnames(zy.pred) <- 1:24
  
  write.table(zy.pred,
            'zy_simulation.txt',
            sep = ',',
            col.names = F,
            append = T)
}

\end{lstlisting}

\item C2-oreder-program-new.m: 在给定4家候选供应商的基础上,实现问题二含概率元素的优化。

\begin{lstlisting}[language={Matlab}]
w = [0.72
    0.6
    0.6 
    0.6];
sigma2 = [10000
          10000
          10000
          10000];
sigma = sqrt(sigma2);
%%
cvx_clear
cvx_begin
    variable x(4,24) 
    expressions mu(24,1) sig(24,1) z(24,1)
    minimize(sum(x(1,:))+1.2*sum(x(2,:)+x(3,:)+x(4,:)))
    subject to
    x>=0;
    
    for j=1:24
        for i=1:4
            z(j)=z(j)+x(i,j)/w(i);
        end
        if(j==1)
            mu(j)=z(j)-28200;
        else
            mu(j)=mu(j-1)+z(j)-28200;
        end
    end
    sig=zeros(24,1);
    for j=1:24
        sig(j)=j.*sum(sigma./w);
    end
    mu-1.6*sig>=0
    
cvx_end

\end{lstlisting}

\item p2.lg4: 问题二订购方案的LINGO求解代码。(对于问题三订购方案的求解代码详见p3.lg4,由于该部分代码除了数据规模、数组大小、文件IO路径等几乎完全一致,p3.lg4不在附录中展示,可详见支撑材料的代码部分)

\begin{lstlisting}
model:
sets:
supplyers/1..4/:;
weeks/1..24/:;
transporters/1..8/:;
rate(transporters,weeks): a;
assignment(supplyers,weeks): S;
decision(supplyers,weeks,transporters): T,ans; 
endsets

data:
a = @ole('C:\Users\HONOR\Desktop\C\data\p2_trans_pred.xlsx','pred_data');
S = @ole('C:\Users\HONOR\Desktop\C\data\p2_supply_prob.xlsx','pred_data');
@ole('C:\Users\HONOR\Desktop\C\res\p2_method_prob.xlsx','pred_data') = ans;
enddata
min = @sum(transporters(k):@sum(supplyers(i): @sum(weeks(j): S(i,j)*T(i,j,k)*a(k,j)))); 
@for(supplyers(i): @for(weeks(j): @sum(transporters(k): T(i,j,k))=1));
@for(transporters(k): @for(weeks(j): @sum(supplyers(i): (T(i,j,k))*S(i,j))<=6000)); 
@for(transporters(k):@for(supplyers(i): @for(weeks(j): @bin(T(i,j,k))))); 
@for(transporters(k): @for(weeks(j): @for(supplyers(i): ans(i,j,k) = T(i,j,k) * S(i,j)))); 
end

\end{lstlisting}

\item C3-order-program-limit6000.m:问题三订购方案的概率优化模型,在问题二的概率优化模型的基础上加入了转运两6000的最大容量限制等。

\begin{lstlisting}[language={Matlab}]

w=zeros(402,1);
w(label=='A')=0.6;
w(label=='B')=0.66;
w(label=='C')=0.72;
sigma=100*ones(402,1);

aindex=find(label=='A');
bindex=find(label=='B');
cindex=find(label=='C');

load 'res.txt'
load 'upbound.txt'

upboundd=upbound*ones(1,24);
shift=res(:,1);
sigma=res(:,2);
sig=zeros(24,1);
for j=1:24
    sig(j)=j.*sum(sigma./w);
end
sig=sig+158.0332; % r.sd in R files 'C3_preprocess'
%%
cvx_clear
cvx_begin
    variable x(402,24) 
    expressions mu(24,1) sig(24,1) z(24,1)
    sa=sum(x(aindex,:));sb=sum(x(bindex,:));sc=sum(x(cindex,:));
    minimize(1.2*sum(sa)+1.1*sum(sb)+sum(sc)+0.1*norm(sc,1)+0.05*norm(sb,1))
    %minimize(1.2*sum(sa)+1.1*sum(sb)+sum(sc))
    subject to
    x>=0;
    x<=upboundd;
    x<=6000;
    for j=1:24
        for i=1:402
            z(j)=z(j)+(x(i,j)+shift(i))/w(i);
        end
        if(j==1)
            mu(j)=z(j)-28200;
        else
            mu(j)=mu(j-1)+z(j)-28200;
        end
    end
    
    mu-1.6*sig>=0
    
cvx_end

\end{lstlisting}

\item p3-end2end.lg4: 问题三端到端的多目标规划求解程序代码。

\begin{lstlisting}
model:
sets:
supplyers/1..402/: u,w;
weeks/1..24/:;
transporters/1..8/:;

rate(transporters,weeks): a;
assignment(supplyers,weeks): S,R,d,b;
decision(supplyers,weeks,transporters): T,ans; 
endsets

data:
alpha = 1;
beta = 1;
gamma = 1;

a = @ole('C:\Users\HONOR\Desktop\C\data\p2_trans_pred.xlsx','pred_data');
@ole('C:\Users\HONOR\Desktop\C\res\p3_method_S.xlsx','S') = S;
@ole('C:\Users\HONOR\Desktop\C\res\p3_method_T.xlsx','T') = ans;

enddata

min = alpha * TransCost + beta * StoreCost + gamma * OrderCost;

OrderCost = @sum(supplyers(i): @sum(weeks(j): S(i,j) * w(i)));
TransCost = @sum(supplyers(i): @sum(weeks(j): S(i,j)));
StoreCost = @sum(supplyers(i): @sum(weeks(j): d(i,j) * u(i)));

@for(supplyers(i): @for(weeks(j): @sum(transporters(k): T(i,j,k) * S(i,j) * (1-a(k,j)) ) = R(i,j)));
@for(supplyers(i): @for(weeks(j): @sum(transporters(k): T(i,j,k))=1));
@for(transporters(k):@for(supplyers(i): @for(weeks(j): @bin(T(i,j,k))))); 
@for(transporters(k): @for(weeks(j): @sum(supplyers(i): (T(i,j,k))*S(i,j))<=6000));
  
@for(supplyers(i): @for(weeks(j)|j#gt#1: d(i,j) = d(i,j-1) + R(i,j) / u(i) - b(i,j)));
@for(weeks(j): @sum(supplyers(i): b(i,j)) = 28200); 
@for(weeks(j): @sum(supplyers(i): d(i,j)) >= 56400);

@for(transporters(k): @for(weeks(j): @for(supplyers(i): ans(i,j,k) = T(i,j,k) * S(i,j)))); 

end
\end{lstlisting}
\item C4-bisection.m: 基于问题三中订购方案的C3的Matlab文件,使用改进版本的概率优化模型,并且利用二分法求上界。

\begin{lstlisting}[language={Matlab}]
w=zeros(402,1);
w(label=='A')=0.6;
w(label=='B')=0.66;
w(label=='C')=0.72;
sigma=100*ones(402,1);

aindex=find(label=='A');
bindex=find(label=='B');
cindex=find(label=='C');


load 'upbound_new.txt'

upbound=upbound_new(:,1)*ones(1,24);
sigma=upbound_new(:,2);
sig=zeros(24,1);
for j=1:24
    sig(j)=j.*sum(sigma./w);
end
sig=sig+158.0332; % r.sd in R files 'C3_preprocess'
%%
bound=76400; % solved
%bound=76500; % infeasible
cvx_clear
cvx_begin
    variable x(402,24) 
    expressions mu(24,1) sig(24,1) z(24,1)
    sa=sum(x(aindex,:));sb=sum(x(bindex,:));sc=sum(x(cindex,:));
    minimize(1.2*sum(sa)+1.1*sum(sb)+sum(sc)+0.2*norm(sc,1)+0.1*norm(sb,1))
    %minimize(1.2*sum(sa)+1.1*sum(sb)+sum(sc))
    subject to
    x>=0;
    x<=upbound;
    sum(x)<=6000*8;
    for j=1:24
        for i=1:402
            z(j)=z(j)+x(i,j)/w(i);
        end
        if(j==1)
            mu(j)=z(j)-bound;
        else
            mu(j)=mu(j-1)+z(j)-bound;
        end
    end
    
    mu-1.6*sig>=0
    
cvx_end

\end{lstlisting}
\item p4.lg4: 问题四订购方案的LINGO求解代码。

\begin{lstlisting}
model:
sets:
supplyers/1..402/:;
weeks/1..24/:;
transporters/1..8/:;
rate(transporters,weeks): a;
assignment(supplyers,weeks): S;
decision(supplyers,weeks,transporters): T; 
endsets

data:
a = @ole('C:\Users\HONOR\Desktop\C\data\p2_trans_pred.xlsx','pred_data');
S = @ole('C:\Users\HONOR\Desktop\C\data\p4_supply_prob.xlsx','pred_data');
@ole('C:\Users\HONOR\Desktop\C\res\p4_method_prob.xlsx','T') = T;
enddata

min = @sum(transporters(k):@sum(supplyers(i): @sum(weeks(j): T(i,j,k)*a(k,j)))); 
@for(supplyers(i): @for(weeks(j): @sum(transporters(k): T(i,j,k)) = S(i,j) ));
@for(transporters(k): @for(weeks(j): @sum(supplyers(i): (T(i,j,k)))<=6000)); 

end

\end{lstlisting}


\item WMA.m: 进行对于已知X,Y列向量,求x处加权滑动均值的Matlab代码。

\begin{lstlisting}[language={Matlab}]

function [y,sigma2] = WMA(X,Y,x)
%WMA 已知X,Y,计算x的对于加权移动平均
% X (N,1), x(n,1), Y(N,1)
N = size(X,1);
n = size(x,1);
% W(n,N)
W = exp(-((X'-x)./500).^2); 
y = W*Y ./ sum(W,2);
sigma2 = sum(W.*((y-Y').^2),2) ./ sum(W,2); %用于试验的方差项;未使用
end

\end{lstlisting}

\item SB\_ proxy.m: 基于WMA.m及已读取的订购量、供应量,拟合第i供应商的订-供曲线。

\begin{lstlisting}[language={Matlab}]

% 拟合第i供应商的订-供曲线
function [S,sigma2] = SB_proxy(i,B)
load datas.mat;
X = Book_matrix(i,:)';
[S,sigma2] = WMA(X,Supply_matrix(i,:)',B);
end

\end{lstlisting}

\item proxy\_ fit.m: 进行具体的试验,取WMA模型下,各供应商可取区间的最大值。最后得到4家供应商并绘图。

\begin{lstlisting}[language={Matlab}]
clc;
load datas.mat
K = 4; %试验得可取出4家公司的数量最优解
Xx = [1:10:6000]'; 

for i = 1:402
    if S_Class(i) == 'B' % 据理论分析,应该选择A或C供应商以最优化
        s_max(i)=-1;p_max(i)=-1;continue
    end
    [s_max(i),I] = max(SB_proxy(i,[1:10:min([6000,S_max(i)])]'));
    p_max(i) = Xx(I);
    [~,sigma2(i)] = SB_proxy(i,p_max(i));
end

[s_max,I] = sort(s_max,'descend');
p_max=p_max(I);
s_max = s_max(1:K);
p_max = p_max(1:K);
sigma2 = sigma2(1:K);
I = I(1:K);

s_max
sigma2
s_max' .* density_vec(I)

for i = 1:K
    figure(i);
    scatter(Book_matrix(I(i),:), ...
            Supply_matrix(I(i),:));
    hold("on");
    plot(Xx,SB_proxy(I(i),Xx));
    scatter(p_max(i),s_max(i));
end
\end{lstlisting}

\item MonteCarlo2.m: 基于已解出的供应量,计算问题2第一部分蒙特卡罗模拟,统计相关数据。

\begin{lstlisting}[language={Matlab}]

% 基于已解出的供应量,计算问题2第一部分蒙特卡罗模拟

load datas.mat
load S1.mat

store = 56400;
tot_cost = 0;

global Wdata;
Wdata = [ ...
2228.722909	2091.471194	2086.687549	2090.396821	2094.586691	2099.442788	2103.946976	2107.057604	2108.235009	2107.664493	2106.234781	2105.22956	2105.817512	2108.409079	2111.887995	2113.042447	2107.463828	2092.775876	2071.379274	2062.778145	2152.900943	2269.523075	1369.960226	2160.012439;
5299.899551	5258.797363	5264.352687	5264.312022	5263.755617	5262.802978	5261.746028	5260.872938	5260.345076	5260.17704	5260.257568	5260.409268	5260.463994	5260.317032	5259.949036	5259.435157	5258.817379	5257.679365	5255.429351	5253.281115	5253.793933	5262.869029	5413.542291	5189.019956;
5299.899551	5258.797363	5264.352687	5264.312022	5263.755617	5262.802978	5261.746028	5260.872938	5260.345076	5260.17704	5260.257568	5260.409268	5260.463994	5260.317032	5259.949036	5259.435157	5258.817379	5257.679365	5255.429351	5253.281115	5253.793933	5262.869029	5413.542291	5189.019956;
5299.899551	5258.797363	5264.352687	5264.312022	5263.755617	5262.802978	5261.746028	5260.872938	5260.345076	5260.17704	5260.257568	5260.409268	5260.463994	5260.317032	5259.949036	5259.435157	5258.817379	5257.679365	5255.429351	5253.281115	5253.793933	5262.869029	5413.542291	5189.019956];
% 已求解的各供货商、各周的供应量(作为均值)。
% 其中行1为C供应商,2:4为A供应商

for time = 1:24
    store;
    
    target = 28200;
    store = store - target;
    if store <0
        error("Can't reach target at a time duration")
    end
    
    [a_in,b_in,c_in] = week_supply(time);
    
    store = store + a_in /0.6 + b_in / 0.66 + c_in / 0.72;
    tot_cost = tot_cost + 1.2*a_in+1.1*b_in+1*c_in;
    
    if store < 56400
        error("Can't reach target at a time duration")
    end
end

%disp("Finished")
tot_cost;

function [a_in,b_in,c_in] = week_supply(w)
    global Wdata;
    
    A_mean = Wdata(2:4,w);
    A_std = [100;100;100];
    C_mean = Wdata(1,w);
    C_std = 100;
    b_in = 0;
    a_in = normrnd(A_mean,A_std);
    c_in = normrnd(C_mean,C_std);
    a_in(a_in<0) = 0; a_in(a_in>6000) = 6000;
    c_in(c_in<0) = 0; c_in(c_in>6000) = 6000;
    a_in = sum(a_in);
    c_in = sum(c_in);
end

\end{lstlisting}

\item run\_MC\_ 2.m: 在此处进行给定次数的MonteCarlo2.m并展示,以及执行另一部分的MonteCarlo2\_T.m

\begin{lstlisting}[language={Matlab}]
MC2_cost = []
MC2_store=[]
for i = 1:10000
    MonteCarlo2;
    MC2_cost=[MC2_cost,tot_cost];MC2_store=[MC2_store,store];
end

mean(MC2_cost)
std(MC2_cost)

MonteCarlo2_T;
\end{lstlisting}

\item MonteCarlo3.m: 进行问题3的蒙特卡罗模拟;在其中进行10000次并统计。
\begin{lstlisting}[language={Matlab}]
load datas.mat
load MC.mat

K = 1;

u = zeros(402,1);
u(S_Class=='A') = 1/0.60;
u(S_Class=='B') = 1/0.66;
u(S_Class=='C') = 1/0.72;

w = zeros(402,1);
w(S_Class=='A') = 1.2;
w(S_Class=='B') = 1.1;
w(S_Class=='C') = 1.0;

C = zeros(402,3);
C(:,1) = S_Class=='A';
C(:,2) = S_Class=='B';
C(:,3) = S_Class=='C';

supply = P3_S; % 402x24 供应量
T = P3_T; % 402x24x8 转运关系
A = permute(MC2_T_sim,[1 3 2]); % (转置后)10000x24x8 带有蒙特卡罗模拟的损耗率

small_count = size(Low_ID,1);
small_std = 158.0332 ^ 0.5;

shift_ex = zeros(402,24);
shift_std_avr = shift_std;
shift_std_avr(Low_ID) = small_std/ small_count^0.5;
shift_std_ex = zeros(402,24);

for i = 1 : 24
    shift_ex(:,i) = shift(:);
    shift_std_ex(:,i) = shift_std_avr;
end

MC3_cost = [];
MC3_store = [];

MC3_err_count = 0;
for s = 1:10000
    As = A(s,:,:); % 1 24 8
    S_std = 100 .* ones(size(supply));
    
    supply_S = normrnd(supply-shift_ex,shift_std_ex);
    supply_S(supply_S<0) = 0;
    supply_S(supply_S>6000) = 6000;
    
    %supply_S = supply;
    
    R = sum((1- 0.01.*As).* T .* reshape(supply_S,[402 24 1]),3);
    R = reshape(R,[402 24]);
    if any(R<0)
        error("R<0")
    end
    In = R'*C; 
    
    % In (24 3) 每周 A,B,C 接受量
    tot_cost = sum(K.*(w .* supply_S)+supply_S,'all');
    store = [56400*0.6 0 0 ];
    %In %
    
    err_flag = 0;
    for week = 1:24
        
        tot_cost = tot_cost + sum(store);
        target = 28200;
        A_used = 0; B_used = 0; C_used = 0;
        if store(3)>0
            C_used = min(target, store(3)./0.72);
            target = target - C_used;
            store(3) = store(3) - C_used*0.72;
        end
        if target > 0 && store(2)>0
            B_used = min(target, store(2)./0.66);
            target = target - B_used;
            store(2) = store(2) - B_used*0.66;
        end
        if target > 0 && store(1)>0
            A_used = min(target, store(1)./0.60);
            target = target - A_used;
            store(1) = store(1) - A_used*0.60;
        end
        
        store = store + In(week,:);
        
        if store' * [1/0.6 1/0.66 1/0.72] < 56400
            disp("store < 56400");
            err_flag = 1;
            continue % 继续执行完当前生产过程
        end
    end
    
    MC3_cost = [MC3_cost tot_cost];
    MC3_store = [MC3_store; store];
    if err_flag == 1
        MC3_err_count = MC3_err_count + 1;
    end
end

disp("Finished")
\end{lstlisting}


\item MonteCarlo2\_ T.m:  进行问题2转运量部分的蒙特卡罗模拟。
\begin{lstlisting}[language={Matlab}]

%已得到转运损耗量波动模拟后,计算问题2第二部分的蒙特卡罗模拟

load MC.mat

for i = 1:10000
    for j = 1:4
        MC2_T_result(i,j,:) = Wdata(j,:) .* sum( reshape(P2_T(j,:,:),[8 24]) .* (1-0.01.*reshape(MC2_T_sim(i,:,:),[8 24])));
    end
end

MC2_T_res = MC2_T_result- reshape(Wdata,[1,4,24]);

\end{lstlisting}

\end{itemize}

\iffalse
除了支撑材料的文件列表和源程序代码外,附录中还可以包括下面内容:
\begin{itemize}
\item 某一问题的详细证明或求解过程;
\item 自己在网上找到的数据;
\item 比较大的流程图;
\item 较繁杂的图表或计算结果
\end{itemize}
\fi 

\end{document}